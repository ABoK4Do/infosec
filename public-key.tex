\chapter{Криптосистемы с открытым ключом}\label{chapter-public-key}
\selectlanguage{russian}

\textbf{Криптосистемой с открытым ключом} (public-key cryptosystem, PKC) называется криптографическое преобразование, использующее два ключа -- открытый и секретный. Пара из \textbf{закрытого}\index{ключ!закрытый} (private key, secret key, SK)\footnote{В контексте криптосистем с открытым ключом можно ещё встретить использование термина <<секретный ключ>>. Мы не рекомендуем использовать данный термин, чтобы не путать с секретным ключом\index{ключ!секретный}, используемым в симметричных криптосистемах} и \textbf{открытого}\index{ключ!открытый} (public key, PK) ключей создается пользователем, который свой закрытый ключ держит в секрете, а открытый ключ делает общедоступным для всех пользователей. Криптографическое преобразование в одну сторону (шифрование) можно выполнить зная только открытый ключ, а в другую (расшифрование) -- только зная секретный ключ. Во многих криптосистемах из закрытого ключа теоретически можно вычислить открытый ключ, однако это является сложной вычислительной задачей.

Если прямое преобразование выполняется открытым ключом, а обратное -- закрытым, то криптосистема называется \textbf{схемой шифрования с открытым ключом}. Все пользователи, зная открытый ключ получателя, могут зашифровать для него сообщение, которое может расшифровать только владелец закрытого ключа.

Если прямое преобразование выполняется закрытым ключом, а обратное -- открытым, то криптосистема называется \textbf{схемой электронной подписи (ЭП)}. Владелец закрытого ключа может \emph{подписать} сообщение, а все пользователи, зная открытый ключ, могут проверить, что подпись была создана только владельцем закрытого ключа и никем другим.

Криптосистемы с открытым ключом снижают требования к каналам связи, которые требуются для передачи данных. В симметричных криптосистемах перед началом связи (перед шифрованием сообщения и его передачей) требуется по защищённому каналу связи передать или согласовать секретный ключ шифрования. Злоумышленник не должен иметь возможность ни прослушать данный канал связи, ни подменить передаваемую информацию (ключ). Криптосистемы с открытым ключом требуют только то, что злоумышленник не должен иметь возможности подменить открытый ключ, который получатель передаёт отправителю для будущего шифрования. Говоря другими словами, криптосистема с открытым ключом, в случае использования открытых и незащищённых каналов связи устойчива к пассивному криптоаналитику\index{криптоаналитик!пассивный}, но всё ещё должна предпринимать меры по защите от активного криптоаналитика\index{криптоаналитик!активный}.

Для предотвращения атак <<человек посередине>> (man-in-the-middle attack)\index{атака!человек посередине} с активным криптоаналитиком\index{криптоаналитик!активный}, который бы подменял открытый ключ получателя во время его передачи будущему отправителю сообщений, используют \textbf{сертификаты открытых ключей}\index{сертификат открытого ключа}. Сертификат представляет собой информацию о соответствии открытого ключа и его владельца, подписанную электронной подписью третьего лица. В корпоративных информационных системах достаточно, если на всю организацию такое лицо, подписывающее сертификаты, будет одно. В этом случае его называют \textbf{доверенным центром сертификации} или \textbf{удостоверяющим центром}. В глобальной сети Интернет для защиты распространения программного обеспечения (например, защиты от подделок в ПО) и проверок сертификатов в протоколах на базе SSL/TLS\index{протокол!SSL/TLS} используется иерархия удостоверяющих центров, рассмотренная в разделе~\ref{section-CAs}. При обмене личными сообщениями и при распространении программного обеспечения с открытым кодом вместо жёсткой иерархии может использоваться \textbf{сеть доверия}\index{сеть доверия}. В сети доверия каждый участник может подписать сертификат любого другого участника. Предполагается, что подписывающий знает лично владельца сертификата и удостоверился о соответствии сертификата владельцу при личной встрече.

Криптосистемы с открытым ключом построены на основе односторонних (однонаправленных) функций c потайным входом. Под \textbf{односторонней} функцией понимают \emph{вычислительную} невозможность вычисления ее обращения: вычисление значения функции $y = f(x)$ при заданном аргументе $x$ является легкой задачей, вычисление аргумента $x$ при заданном значении функции $y$ -- трудной задачей.

Односторонняя функция $y = f(x,K)$ с \textbf{потайным входом}\index{функция!с потайным входом} $K$ определяется как функция, которая легко вычисляется при заданном $x$, и аргумент $x$ которой можно легко вычислить из $y$, если известен <<секретный>> параметр $K$, и вычислить невозможно, если параметр $K$ неизвестен.

Примером подобной функции является возведение в степень по модулю составного числа $n$:
	\[ c = f \left( m \right) = m ^ e \mod n.\]

Для того, чтобы быстро вычислить обратную функцию
	\[ m = f^{-1} \left( c \right) = \sqrt[e]{c} \mod n, \]
её можно представить в виде
	\[ m = y^{d} \mod m,\]
где
	\[ d = e^{-1} \mod \varphi \left( n \right). \]

В последнем выражении $\varphi \left( n \right)$ -- это функция Эйлера\index{функция!Эйлера}. В качестве <<потайной дверцы>> или секрета можно рассматривать или непосредственно само число <<$d$>>, или значение $\varphi \left( b \right)$. Последнее можно быстро найти только в том случае, если известно разложение числа $n$ на простые сомножители. Именно эта функция с потайной дверцей лежит в основе криптосистемы RSA\index{криптосистема!RSA}.

Необходимые математические основы модульной арифметики, групп, полей и простых чисел приведены в Приложении \ref{chap:discrete-math}.

\section{Криптосистема RSA}\index{криптосистема!RSA|(}
\selectlanguage{russian}

\subsection{Шифрование}\index{шифр!RSA|(}

В 1978 г. Рональд Рив\'{е}ст, Ади Шамир и Леонард Адлеман (\langen{Ronald Linn Rivest, Adi Shamir, Leonard Max Adleman}, \cite{RSA:1978}) предложили алгоритм, обладающий рядом интересных для криптографии свойств. На его основе была построена первая система шифрования с открытым ключом, получившая название по первым буквам фамилий авторов -- система RSA.

Рассмотрим принцип построения криптосистемы шифрования RSA с открытым ключом.

\begin{enumerate}
    \item \textbf{Создание пары из закрытого и открытого ключей.}
        \begin{enumerate}
            \item Случайно выбрать большие простые\index{число!простое} различные числа $p$ и $q$, для которых $\log_2 p \simeq \log_2 q > 1024$ бит\footnote{Случайный выбор больших простых чисел не является простой задачей. См. раздел~\ref{section-pseudo-primes-generation} в приложении.}.
            \item Вычислить произведение $n = pq$.
            \item Вычислить функцию Эйлера\index{функция!Эйлера}\footnote{См. раздел~\ref{section-group-multiplicative} в приложении.} $\varphi(n) = (p-1)(q-1)$.
            \item Выбрать случайное целое число $e \in [3, \varphi(n)-1]$, взаимно простое с $\varphi(n)$: $~ \gcd(e, \varphi(n)) = 1$.
            \item Вычислить число $d$ такое, что $d \cdot e = 1 \mod \varphi(n)$.
            \item Закрытым ключом будем называть числа $n$ и $d$, открытым ключом\footnote{Некоторые авторы считают некорректным включать число $n$ в состав закрытого ключа, так как оно уже входит в открытый. Авторы настоящего пособия включают число $n$ в состав закрытого ключа, что в результате позволяет в дальнейшем использовать для расшифрования и создания электронной подписи данные \emph{только} из закрытого ключа, не прибегая к <<помощи>> данных из открытого ключа.} -- $n$ и $e$.
        \end{enumerate}

    \item \textbf{Шифрование с использованием открытого ключа}
        \begin{enumerate}
            \item Сообщение представляют целым числом $m \in [1, n-1]$.
            \item Шифротекст вычисляется как
                \[ c = m^e \mod n. \]
                Шифротекст -- также целое число из диапазона $[1, n-1]$.
        \end{enumerate}
    \item \textbf{Расшифрование с использованием закрытого ключа}

        Владелец закрытого ключа вычисляет
                \[ m = c^d \mod n. \]
\end{enumerate}

Покажем корректность схемы шифрования RSA. В результате расшифрования шифротекста $c$ (полученного путём шифрования открытого текста $m$) легальный пользователь имеет:
\[\begin{array}{ll}
    c^{d} & = m^{ed} \mod p = \\
          & = m^{ 1 + \alpha_1 \cdot \varphi(n)} \mod p = \\
          & = m^{ 1 + \alpha_1 \cdot ( p - 1 ) ( q - 1 )} \mod p = \\
          & = m^{ 1 + \alpha_2 \cdot ( p - 1 )} \mod p = \\
          & = m \cdot m^{\alpha_2 \cdot ( p - 1 )} \mod p. \\
\end{array}\]

Если $m$ и $p$ являются взаимно простыми, то из малой теоремы Ферма\index{теорема!Ферма малая} следует, что:
	\[m^{\left( p - 1 \right)} = 1 \mod p,\]
\[\begin{array}{ll}
	c^{d} & = m \cdot m^{\alpha_2 \cdot \left( p - 1 \right)} = \\
	      & = m \cdot \left( m^{\left(p - 1\right)} \right)^{\alpha_2} = \\
	      & = m \cdot 1^{\alpha_2} = \\
	      & = m \mod p.
\end{array}\]

Если же $m$ и $p$ не являются взаимно простыми, то есть $p$ является делителем $m$ (помним, что $p$ -- простое число), то $m = 0 \mod p$ и $c^{d} = 0 \mod p$.

В результате, для любых $m$ верно, что $c^{d} = m \mod p$. Аналогично доказывается, что $c^{d} = m \mod q$. Из китайской теоремы об остатках\index{теорема!китайская об остатках} (см. раздел~\ref{section-chinese-remainder-theorem} в приложении) следует:

\[\begin{cases}
	n = p \cdot q, \\
	c^{d} = m \mod p, \\
	c^{d} = m \mod q,
\end{cases} \Rightarrow c^{d} = m \mod n.\]

\example Создание ключей, шифрование и расшифрование в криптосистеме RSA.

\begin{enumerate}
    \item Генерирование параметров.
        \begin{enumerate}
            \item Выберем числа $p=13, q=11, n = 143$.
            \item Вычислим $\varphi(n) = (p-1)(q-1) = 12 \cdot 10 = 120$.
            \item Выберем $e=23: ~ \gcd(e, \varphi(n))=1, ~ e \in [3, 119]$.
            \item Найдём $d = e^{-1} \mod \varphi(n) = 23^{-1} \mod 120 = 47$.
            \item Открытый и закрытый ключи:
                \[ \PK = (e:23, n:143), ~ \SK = (d:47, n:143). \]
        \end{enumerate}
    \item Шифрование.
        \begin{enumerate}
            \item Пусть сообщение $m = 22 \in [1, n-1]$.
            \item Вычислим шифротекст:
                \[ c = m^e = 22^{23} \mod 143 = 55 \mod 143. \]
        \end{enumerate}
    \item Расшифрование.
        \begin{enumerate}
            \item Полученный шифротекст $c = 55$.
            \item Вычислим открытый текст:
                \[ m = c^d = 55^{47} \mod 143 = 22 \mod 143. \]
        \end{enumerate}
\end{enumerate}
\exampleend

%Рассмотрим её основные положения на примере криптосистемы с открытым ключом.
%Приведём общую схему алгоритма RSA.
%$C_i=M_{i}^{E_k}(mod N_j)$
%$N_j=P_{j}Q_{j}$
%$M_i=C_{i}^{D_k}(mod N_j)$
%$E_k\neq D_k$
%Вычислить $E_k$ из $D_k$ при длине блока сообщения $L_{блока} > L_{дополнения}$ можно только с экспоненциальной сложностью. $E_k D_K=1(mod \varphi(N_j))$
%Данное сравнение не даёт единственного решения. Решение данного сравнения и можно свести к следующему уравнению:
%$ax+by=1$
%$E_k D_k=k \varphi(N_j)+1$
%$1\leq E_k D_k <\varphi(N_j)$
%$\varphi(N_j)(-k)+ E_k D_k=1$
%Стандарт ISO X.509 определяет требования по реализации алгоритма RSA, в частности, требования к общесистемным параметрам и ключам, методы распространения сертификатов ключей и ключевых параметров, а также порядок ввода их в действие и многое другое.
\index{шифр!RSA|)}

\subsection{Электронная подпись}\index{электронная подпись!RSA|(}

Предположим, что пользователь $A$ не шифрует свои сообщения, но хочет посылать их в виде открытых текстов с подписью. Для этого надо создать электронную подпись (ЭП). Это можно сделать, используя систему RSA. При этом должны быть выполнены следующие требования:
\begin{itemize}
    \item вычисление подписи от сообщения является вычислительно лёгкой задачей;
    \item фальсификация подписи при неизвестном закрытом ключе -- вычислительно трудная задача;
    \item подпись должна быть проверяемой открытым ключом.
\end{itemize}

Создание параметров ЭП RSA производится так же, как и для схемы шифрования RSA. Пусть $A$ имеет закрытый ключ $\SK = (n, d)$, а получатель (проверяющий) $B$ -- открытый ключ $\PK = (e,n)$ пользователя $A$.

\begin{enumerate}
    \item $A$ вычисляет подпись сообщения $m \in [1,n-1]$ как
        \[ s = m^{d} \mod n \]
        на своём закрытом ключе $\SK$.
    \item $A$ посылает $B$ сообщение в виде $(m, s)$, где $m$ -- открытый текст, $s$ -- подпись.
    \item $B$ принимает сообщение $(m, s)$, возводит $s$ в степень $e$ по модулю $n$ ($e, n$ -- часть открытого ключа). В результате вычислений $B$ получает открытый текст
        \[ s^{d} \mod n = \left( m^{d} \mod n \right)^{e} \mod n = m. \]
    \item Сравнивает полученное значение с первой частью сообщения. При полном совпадении подпись принимается.
\end{enumerate}
Недостаток этой системы создания ЭП состоит в том, что подпись $m^{d} \mod n$ имеет большую длину, равную длине открытого сообщения $m$.

Для уменьшения длины подписи применяется другой вариант процедуры: вместо сообщения $m$ отправитель подписывает $h(m)$, где $h(x)$ -- известная криптографическая хэш-функция. Модифицированная процедура состоит в следующем.

\begin{enumerate}
    \item $A$ посылает $B$ сообщение в виде $(m, s)$, где $m$ -- открытый текст,
        \[ s = h(m)^d \mod n \]
        -- подпись.
    \item $B$ принимает сообщение $(m, s)$, вычисляет хэш $h(m)$ и возводит подпись в степень
        \[ h_1 = s^e \mod n. \]
    \item $B$ сравнивает значения $h(m)$ и $h_1$. При равенстве
        \[ h(m) = h_1 \]
        подпись считается подлинной, при неравенстве -- фальсифицированной.
\end{enumerate}


\example Создание и проверка электронной подписи в криптосистеме RSA.

\begin{enumerate}
    \item Генерирование параметров.
        \begin{enumerate}
            \item Выберем $p=13, q=17, n = 221$.
            \item Вычислим $\varphi(n) = (p-1)(q-1) = 12 \cdot 16 = 192$.
            \item Выберем $e=25: ~ \gcd(e = 25, \varphi(n) = 192) = 1, \\
                e \in [3, \varphi(n) - 1 = 191]$.
            \item Найдём $d = e^{-1} \mod \varphi(n) = 25^{-1} \mod 192 = 169$.
            \item Открытый и закрытый ключи:
                \[ \PK = (e:25, n:221), ~ \SK = (d:169, n:221). \]
        \end{enumerate}
    \item Подписание.
        \begin{enumerate}
            \item Пусть хэш сообщения $h(m) = 12 \in [1, n-1]$.
            \item Вычислим ЭП:
                \[ s = h^d = 12^{169} = 90 \mod 221. \]
        \end{enumerate}
    \item Проверка подписи.
        \begin{enumerate}
            \item Пусть хэш полученного сообщения $h(m) = 12$, полученная подпись $s = 90$.
            \item Выполним проверку:
                \[ h_1 = s^e = 90^{25} = 12 \mod 221, ~~ h_1 = h. \]
                Подпись верна.
        \end{enumerate}
\end{enumerate}

\index{электронная подпись!RSA|)}

\subsection{Семантическая безопасность шифров}

\textbf{Семантически безопасной}\index{криптосистема!семантически-безопасная} называется криптосистема, для которой вычислительно невозможно извлечь любую информацию из шифротекстов, кроме длины шифротекста. Алгоритм RSA не является семантически безопасным. Одинаковые сообщения шифруются одинаково, и следовательно применима атака на различение сообщений.

Кроме того, сообщения длиной менее $\frac{k}{3}$ бит, зашифрованные на малой экспоненте $e=3$, \emph{дешифруются} нелегальным пользователем извлечением обычного кубического корня.

В приложениях RSA используется только в сочетании с рандомизацией\index{рандомизация шифрования}. В стандарте PKCS\#1 RSA Laboratories описана схема рандомизации перед шифрованием OAEP-RSA (Optimal Asymmetric Encryption Padding). Примерная схема:
\begin{enumerate}
    \item Выбирается случайное $r$.
    \item Для открытого текста $m$ вычисляется
        \[ x = m \oplus H_1(r), ~ y = r \oplus H_2(x), \]
        где $H_1$ и $H_2$ -- криптографические хэш-функции.
    \item Сообщение $M = x ~\|~ y$ далее шифруется RSA.
\end{enumerate}
Восстановление $m$ из $M$ при расшифровании:
    \[ r = y \oplus H_2(x), ~ m = x \oplus H_1(r). \]

В модификации OAEP+ $x$ вычисляется как
    \[ x = (m \oplus H_1(r)) \| H_3(m \| r). \]

В описанной выше схеме ЭП под $m$ понимается хэш открытого текста, вместо шифрования выполняется подписание, вместо расшифрования -- проверка подписи.


\subsection{Выбор параметров и оптимизация}

\subsubsection{Выбор экспоненты $e$}

В случайно выбранной экспоненте $e$ c битовой длиной $k = \lceil \log_2 e \rceil$ одна половина битов в среднем равна 0, другая -- 1. При возведении в степень $m^e \mod n$ по методу <<возводи в квадрат и перемножай>> получится $k-1$ возведений в квадрат и в среднем
 $\frac{1}{2}(k-1)$ умножений.

Если выбрать $e$, содержащую малое число единиц в двоичной записи, то число умножений уменьшится до числа единиц в $e$.

Часто экспонента $e$ выбирается \emph{малым} \emph{простым} числом и/или содержащим малое число единиц в битовой записи для ускорения шифрования или проверки подписи, например:
\[
    \begin{array}{l}
        3 = [11]_2, \\
        17 = 2^4+1 = [10001]_2, \\
        257 = 2^8+1 = [100000001]_2, \\
        65537 = 2^{16}+1 = [10000000000000001]_2.
    \end{array}
\]

%Время шифрования или проверки подписи для малых экспонент становится $O(k^2)$ вместо $O(k^3)$, то есть в сотни раз быстрее для 1000-битовых чисел.


\subsubsection{Ускорение~шифрования по~китайской~теореме об~остатках}

Возводя $m$ в степень $e$ отдельно по $\mod p$ и $\mod q$ и применяя китайскую теорему об остатках\index{теорема!китайская об остатках} (Chinese remainder theorem, CRT), можно быстрее выполнить шифрование.

Однако ускорение шифрования в криптосистеме RSA через CRT может привести к уязвимостям в некоторых применениях, например в смарт-картах.

\example
Пусть $c = m^e \mod n$ передаётся на расшифрование на смарт-карту, где вычисляется
\[ \begin{array}{c}
    m_p = c^d \mod p, \\
    m_q = c^d \mod q, \\
    m = m_p q (q^{-1} \mod p) + m_q p (p^{-1} \mod q) \mod n. \\
\end{array} \]
Криптоаналитик внешним воздействием может вызвать сбой во время вычисления $m_p$ (или $m_q$), в результате получится $m_p'$ и $m'$ вместо $m$. Зная $m_p'$ и $m'$, криптоаналитик находит разложение числа $n$ на множители $p,q$:
    \[ \gcd(m' - m, ~ n) = \gcd( (m_p' - m) q (q^{-1} \mod p), ~ pq) = q. \]
\exampleend


\subsubsection{Длина ключей}

В 2005 году было разложено 663-битовое число вида RSA. Время разложения в эквиваленте составило 75 лет вычислений одного ПК. Самые быстрые алгоритмы факторизации -- субэкспоненциальные\index{задача!факторизации}. Минимальная рекомендуемая длина модуля $n$ = 1024 бита, но лучше использовать 2048 или 4096 бит.

В июле 2012 года NIST опубликовала отчёт~\cite{NIST:SP800:57}, который включал в себя таблицу сравнения надёжности ключей с разной длиной для криптосистем, относящихся к разным классам. Таблица была составлена согласно как известным на тот момент атакам на классы криптосистем, так и на конкретные шифры (см.~\ref{table:aesrsakeycompare}).
	
\begin{table}[h]
\begin{tabular}{|c|c|c|c|}
\hline
\multicolumn{1}{|p{0.2\linewidth}|}{бит безопасности} & \multicolumn{1}{|p{0.2\linewidth}|}{пример симметричного шифра} & \multicolumn{1}{|p{0.2\linewidth}|}{$\log_2 (n)$ для RSA\tablefootnote{Сравнимая по предоставляемой безопасности битовая длина произведения $n$ простых чисел $p$ и $q$ для криптосистем, основанных на сложности задачи факторизации числа $n$ на простые множители $p$ и $q$, в том числе RSA.}} & \multicolumn{1}{|p{0.2\linewidth}|}{$\log_2 (\| \group{G} \| )$ для эллиптических кривых\tablefootnote{Сравнимая по предоставляемой безопасности битовая длина количества элементов $\|\group{G}\|$ в выбранной циклической подгруппе $\group{G}$ группы точек $\group{E}$ эллиптической кривой для криптосистем, основанных на сложности дискретного логарифма в группах точек эллиптических кривых над конечными полями (см.~\ref{section-elliptic-curve-cryptosystems})}} \\
\hline
80	& 	2TDEA	&	1024	&	160--223	\\
112	& 	3TDEA	&	2048	&	224--255	\\
128	& 	AES-128	&	3072	&	256--383	\\
192	& 	AES-192	&	7680	&	384--511	\\
256	& 	AES-256	&	15360	&	512+	\\
\hline
\end{tabular}
\caption{Сравнимые длины ключей блочных симметричных шифров и ключевых параметров асимметричных шифров~\cite{NIST:SP800:57}}\label{table:aesrsakeycompare}
\end{table}

В приложении~\ref{section-modular-arithmetic} показано, что битовая сложность (количество битовых операций) вычисления произвольной степени $a^b \mod n$ является кубической $O(k^3)$, а возведения в квадрат $a^2 \mod n$ и умножения $a b \mod n$ -- квадратичной $O(k^2)$, где $k$ -- битовая длина чисел $a,b,n$.

%Увеличение длины модуля $n$ в 2 раза увеличивает время возведения в степень в $2^3$ раз для большой экспоненты $e$, а для маленькой экспоненты -- в $2^2$ раза.

\index{криптосистема!RSA|)}


\section{Криптосистемы Эль-Гамаля}
\selectlanguage{russian}
\index{криптосистема!Эль-Гамаля}

\subsection[Шифрование]{Шифрование Эль-Гамаля}

Эта система шифрования с открытым ключом опубликована Эль-Гамалем (El-Gamal) в 1985 году. Рассмотрим принципы ее построения.

Пусть имеется мультипликативная группа $\Z_p^* = \{1, 2, \dots, p-1\}$, где $p$ -- большое простое число, содержащее 1024 двоичных разряда. Существует целое число $g$, называемое примитивным элементом, который порождает все ненулевые числа группы, причем $1 < g < p-1$.

    \[ g\mod p, ~~ g^2\mod p, ~~ \dots, ~~ g^{p-1} = 1\mod p \]

Выберем целое число $x$ в интервале $1 \le x \le p-1$. Вычислим
    \[ y = g^x \mod p. \]
Известно, что в конечном поле функция $y(x)$ -- вычислительно однонаправленная.

Задачей \textbf{дискретного логарифмирования}\index{задача!дискретного логарифмирования} в мультипликативной группе $\Gr$ называется нахождение $x$ по заданным элементам $a,b \in \Gr, ~ b = a^x$. Для групп большого размера $2^{150}$--$2^{1000}$ при выборе элемента $a$ генератором группы или подгруппы большого порядка дискретный логарифм известными алгоритмами не вычислим за доступное время, все известные алгоритмы -- неполиномиальные.

Процедура шифрования криптосистемы  Эль-Гамаля состоит из следующих операций.

\begin{enumerate}
    \item \textbf{Создание пары из секретного и открытого ключей стороной $A$.}
        \begin{enumerate}
            \item $A$ выбирает простое случайное число $p$.
            \item Выбирает генератор $g$ (в программных реализациях алгоритма генератор часто выбирается малым числом, например $g = 2 \mod p$).
            \item Выбирает $x \in [1, p-1]$ с помощью генератора случайных чисел.
            \item Вычисляет $y=g^{x}\mod p$.
            \item Создает секретный и открытые ключи $\SK$ и $\PK$:
                \[ \SK = (x), ~ \PK = (p, g, y). \]
                Криптостойкость задается битовой длиной параметра $p$.
        \end{enumerate}
    \item \textbf{Шифрование на открытом ключе стороной $B$.}
        \begin{enumerate}
            \item $B$ извлекает открытый ключ $\text{PK} = (p, g, y)$ из директории стороны $A$.
            \item Сообщение представляется числом $m \in [1,p-1]$.
            \item Выбирает случайное число $r \in [1, p-1]$ и вычисляет
                \[ \begin{array}{l}
                    a = g^r \mod p, \\
                    b = m y^r \mod p.
                \end{array} \]
            \item Создает шифрованное сообщение в виде
                \[ c = (a,b) \]
                и посылает стороне $A$.
        \end{enumerate}
    \item \textbf{Расшифрование на секретном ключе стороной $A$.}
        \begin{enumerate}
            \item Используя секретный ключ $x$, $A$ вычисляет
                \[ m = \frac{b}{a^x} \mod p. \]
            \item Расшифрование корректно, так как
                \[ \begin{array}{l}
                    \frac{b}{a^x} = \frac{m y^r}{g^{rx}} = m \mod p, \\
                    m \mod p \equiv m.
                \end{array} \]
        \end{enumerate}
\end{enumerate}


\subsubsection{Пример системы}

\begin{enumerate}
    \item Генерирование параметров.
        \begin{enumerate}
            \item Выберем $p=41$.
            \item Группа $\Z_p^*$ циклическая, найдем генератор (примитивный элемент). Порядок группы
                \[ |\Z_p^*| = \varphi(p) = p-1 = 40. \]
                Делители 40: 2, 4, 5, 8, 10, 20. Элемент группы является примитивным, если все его степени, соответствующие делителям порядка группы, не сравнимы с 1. Из табл. \ref{tab:elgamal-generator-search} видно, что число $g = 6$ является генератором всей группы.
                \begin{table}[h!]
                    \centering
                    \caption{Поиск генератора в циклической группе $\Z_{41}^*$. Элемент 6 -- генератор\label{tab:elgamal-generator-search}}
                    \resizebox{\textwidth}{!}{ \begin{tabular}{|c|c|c|c|c|c|c|c|c|}
                        \hline
                        \multirow{2}{*}{Элемент} & \multicolumn{7}{|c|}{Степени} & \multirow{2}{*}{Порядок элемента} \\
                        \cline{2-8}
                                & 2   & 4   & 5   & 8  & 10 & 20 & 40 & \\
                        \hline
                        2       & 4   & 16  & -9  & 10 & -1 & 1  &    & 20 \\
                        3       & 9   & -1  & -3  & 1  &    &    &    & 8 \\
                        5       & -16 & 10  & 9   & 18 & -1 & 1  &    & 20 \\
                        6       & -5  & -16 & -14 & 10 & -9 & -1 & 1  & 40 \\
                        \hline
                    \end{tabular} }
                \end{table}
            \item Выберем случайное $x = 19 \in [1, p-1]$.
            \item Вычислим
                \[ \begin{array}{ll}
                    y & = g^x \mod p = \\
                    & = 6^{19} \mod 41 = \\
                    & = 6^{1 + 2 + 4 \cdot 0 + 8 \cdot 0 + 16} \mod 41 = \\
                    & = 6^1 \cdot 6^2 \cdot 6^{4 \cdot 0} \cdot 6^{8 \cdot 0} \cdot 6^{16} \mod 41 = \\
                    & = 6 \cdot (-5) \cdot (-16)^0 \cdot 10^0 \cdot 18 \mod 41 = \\
                    & = -7 \mod 41.
                \end{array} \]
            \item Открытый и секретные ключи:
                \[ \PK = (p:41, g:6, y:-7), ~ \SK = (x:19). \]
        \end{enumerate}
    \item Шифрование.
        \begin{enumerate}
            \item Пусть сообщением является число $m = 3 \in \Z_p^*$.
            \item Выберем случайное число $r = 25 \in [1, p-1]$.
            \item Вычислим
                \[ \begin{array}{l}
                    a = g^r \mod p = 6^{25} \mod 41 = 14 \mod 41, \\
                    b = m y^r \mod p = 3 \cdot (-7)^{25} \mod 41 = -9 \mod 41.
                \end{array} \]
            \item Шифротекстом является пара чисел
                \[ c = (a:14, ~ b:-9). \]
        \end{enumerate}
    \item Расшифрование.
        \begin{enumerate}
            \item Пусть получен шифротекст
                \[ c = (a:14, ~ b:-9). \]
            \item Вычислим открытый текст как
                \[ \begin{array}{ll}
                    m & = \frac{b}{a^x} \mod p = \\
                    & = -9 \cdot (14^{-1})^{19} \mod 41 = \\
                    & = -9 \cdot 3^{19} \mod 41 = \\
                    & = -9 \cdot (-14) \mod 41 = \\
                    & = 3 \mod 41. \\
                \end{array} \]
        \end{enumerate}
\end{enumerate}


\subsection[Электронная подпись]{Электронная подпись Эль-Гамаля}

Криптосистема Эль-Гамаля, как и криптосистема RSA\index{криптосистема!RSA}, может быть использована для создания ЭП.

По-прежнему имеются два пользователя $A$ и $B$ и незащищенный канал связи между ними. Пользователь $A$  хочет подписать свое открытое сообщение $m$  для того, чтобы пользователь $B$ мог убедиться, что именно $A$ подписал сообщение.

Пусть $A$ имеет секретный ключ $\SK = (x)$, открытый ключ $\PK = (p,g,y)$ (полученные так же, как и в системе шифрования Эль-Гамаля) и хочет подписать открытое сообщение. Обозначим подпись $S(m)$.

Для создания подписи $S(m)$ пользователь $A$ выполняет следующие операции:
\begin{itemize}
    \item вычисляет значение криптографической хэш-функции  $h(m) \in [0,p-2]$, от своего открытого сообщения $m$;
    \item выбирает случайное число $r, ~ \gcd(r, p-1)=1$;
    \item используя открытый ключ, вычисляет
        \[ \begin{array}{l}
            a = g^r \mod p, \\
            b = \frac{h(m) - xa}{r} \mod (p-1); \\
        \end{array} \]
    \item создает подпись в виде двух чисел
        \[ S(m) = (a, b) \]
        и посылает сообщение с подписью $(m, S(m))$.
\end{itemize}

Получив сообщение, $B$ осуществляет проверку подписи, выполняя следующие операции:
\begin{itemize}
    \item по известному сообщению $m$ вычисляет значение хэш-функции $h(m)$;
    \item вычисляет
        \[ \begin{array}{l}
            f_1 = g^{h(m)} \mod p, \\
            f_2 = y^a a^b \mod p; \\
        \end{array} \]
    \item сравнивает значения $f_1$ и $f_2$, если
        \[ f_1 = f_2, \]
        то подпись подлинная, в противном случае -- фальсифицированная (или случайно испорченная).
\end{itemize}

Покажем, что проверка подписи корректна. По малой теореме Ферма получаем
\[ \begin{array}{ll}
    f_1 & = g^{h(m)} \mod p = \\
    & \\
    & = g^{h(m) \mod (p-1)} \mod p; \\
\end{array} \] \[ \begin{array}{ll}
    f_2 & = y^a a^b \mod p = \\
    & = \underbrace{\left( g^x \right)^a}_{y^a} \cdot
        \underbrace{\left( g^r \mod p \right)^{\frac{h(m) - xa}{r} \mod (p-1)}}_{a^b} \mod p = \\
    & \\
    & = g^{xa \mod (p-1)} ~\cdot~ g^{h(m) - xa \mod (p-1)} \mod p = \\
    & = g^{h(m) \mod (p-1)} \mod p = \\
    & = f_1.
\end{array} \]

\subsubsection{Пример системы}

\begin{enumerate}
    \item Генерирование параметров.
        \begin{enumerate}
            \item Выберем $p=41$.
            \item Выберем генератор $g=6$ в группе $\Z_{41}^*$.
            \item Выберем случайное $x = 19 \in [1, p-1]$.%, ~ \gcd(x, p-1) = 1$.
            \item Вычислим
                \[ \begin{array}{ll}
                    y & = g^x \mod p = \\
                    & = 6^{19} \mod 41 = \\
                    & = 6^{1 + 2 + 4 \cdot 0 + 8 \cdot 0 + 16} \mod 41 = \\
                    & = 6 \cdot (-5) \cdot (-16)^0 \cdot 10^0 \cdot 18 \mod 41 = \\
                    & = -7 \mod 41. \\
                \end{array} \]
            \item Открытый и секретные ключи:
                \[ \PK = (p:41, g:6, y:-7), ~ \SK = (x:19). \]
        \end{enumerate}
    \item Подписание.
        \begin{enumerate}
            \item От сообщения $m$ вычисляется хэш $h = H(m)$. Пусть хэш $h  = 3 \in [0, p-2]$.
            \item Выберем случайное $r = 9 \in [1, p-2]$: \\
                $\gcd(r=9, p-1 = 40) = 1$.
            \item Вычислим
                \[ \begin{array}{ll}
                    a & = g^r \mod p = \\
                      & = 6^9 \mod 41 = 19 \mod 41, \\
                    b & = \frac{h - xa}{r} \mod (p-1) = \\
                      & = (3 - 19 \cdot 19) \cdot 9^{-1} \mod 40 = \\
                      & = 2 \cdot 9 \mod 40 = 18 \mod 40. \\
                \end{array} \]
            \item Подпись
                \[ s = (a:19, b:18). \]
        \end{enumerate}
    \item Проверка подписи.
        \begin{enumerate}
            \item Для полученного сообщения находится хэш $h = H(m) = 3$. Пусть полученная подпись к нему имеет вид
                \[ s = (a:19, b:18). \]
            \item Вычислим
                \[ \begin{array}{ll}
                    f_1 & = g^h \mod p = \\
                        & = 6^3 \mod 41 = 11 \mod 41, \\
                    f_2 & = y^a a^b \mod p = \\
                        & = (-7)^{19} \cdot 19^{18} \mod 41 = 11 \mod 41. \\
                \end{array} \]
            \item Проверим равенство $f_1$ и $f_2$. Подпись верна, так как
                \[ f_1 = f_2 = 11. \]
        \end{enumerate}
\end{enumerate}


\subsection[Криптостойкость]{Криптостойкость систем \protect\\ Эль-Гамаля}

Пусть дано уравнение $y=g^{x} \mod p$, требуется определить $x$ в интервале $0<x<p-1$. Задача называется \textbf{дискретным логарифмированием}\index{задача!дискретного логарифмирования}.

Рассмотрим возможные способы нахождения неизвестного числа $x$. Начнем с перебора различных значений $x$ из интервала $0<x<p-1$ и проверки равенства $y=g^{x} \mod p$. Число попыток в среднем равно $\frac{p}{2}$, при $p=2^{1024}$ это число равно $2^{1023}$, что на практике не осуществимо.

Другой подход предложен советским математиком Гельфондом\index{алгоритм!Гельфонда} безотносительно к криптографии. Он состоит в следующем.
Вычислим $S=\lceil\sqrt{p-1}\rceil $, где скобки означают ближайшее (сверху) целое от числа $\sqrt{p-1} $.

Представим искомое число $x$   в следующем виде

\begin{equation}
    x=x_{1} S+x_{2},
    \label{S}
\end{equation}

где $x_{1}$ и $x_{2}$ -- целые неотрицательные числа,
    \[ x_{1} \le S-1, ~ x_{2} \le S-1. \]
Такое представление является однозначным.

Вычислим и занесем в таблицу следующие $S$  чисел:
    \[ g^{0} \mod p, ~~ g^{1} \mod p, ~~ g^{2} \mod p, ~~ \dots, ~~ g^{S-1} \mod p. \]
Вычислим $g^{-S} \mod p$ и также занесем в таблицу.

\begin{center} \begin{tabular}{|l|c|c|c|c|c|c|}
    \hline
    $\lambda $ & 0 & 1 & 2 & \dots & $S-1$ & $-S$ \\
    \hline
    $g^{\lambda} \mod p$ & $g^{0}$ & $g^{1}$ & $g^{2}$ & \dots & $g^{S-1}$ & $g^{-S}$ \\
    \hline
\end{tabular} \end{center}

Для решения уравнения \ref{S} используем перебор значений $x_{1}$.
\begin{enumerate}
    \item  Предположим, что $x_{1} = 0$. Тогда $x = x_{2}$.  Если число $y = g^{x_{2}} \mod p$ содержится в таблице, то  находим его и выдаем результат: $x=x_{2} $. Задача решена. В противном случае переходим к пункту 2.
    \item  Предположим, что $x_{1} =1$. Тогда $x=S+x_{2} $ и $y=g^{S+x_{2}} \mod p$. Вычисляем $yg^{-S} \mod p=g^{x_{2}} \mod p$. Задача сведена к предыдущей: если $g^{x_{2} } \mod p$ содержится в таблице, то в таблице находим число $x_{2} $ и выдаем результат $x$: $x=S+x_{2} $.
    \item  Предположим, что $x_{1} =2$. Тогда $x=2S+x_{2} $ и $y=g^{2S+x_{2} } \mod p$. Если число $yg^{-2S} \mod p=g^{x_{2} } \mod p$ содержится в таблице, то находим число $x_{2}$ и выдаем результат: $x = 2S + x_{2}$.
     \item  Пробегая все возможные значения, доберемся, в худшем случае, до $x_{1} =S-1$. Тогда $x=(S-1)S+x_{2} $ и $y = g^{(S-1)S+x_{2} } \mod p$. Если число $yg^{-(S-1)S} \mod p=g^{x_{2}} \mod p$ содержится в таблице, то  находим его и выдаем результат: $x=(S-1)S+x_{2}$.
\end{enumerate}

Легко проверить, что с помощью построенной таблицы мы проверили все возможные значения $x$. Максимальное число умножений равно $2S \approx 2\sqrt{p-1} =2\times 2^{512} $, что для практики очень велико.  Тем самым проблему достаточной криптостойкости этой системы можно было бы считать решенной. Однако это не верно, так как числа $p-1$ являются составными. Если  $p-1$ можно разложить на маленькие множители, то криптоаналитик может применить процедуру, подобную процедуре Гельфонда, по взаимно простым делителям  $p-1$  и найти секрет. Пусть  $p-1=st$. Тогда элемент $g^s$ образует подгруппу порядка $t$ и наоборот. Теперь, решая уравнение $y^s=(g^s)^a\mod p$, находим вычет $x=a\mod t$. Поступая аналогично, находим $x=b\mod s$ и по Китайской теореме об остатках находим $x$.

Несколько позже подобный метод ускоренного решения уравнения \ref{S} был предложен Шенксом (Shanks)и Хеллманом (Hellman). В англоязычной технической литературе он получил название алгоритма Шенкcа.

Пусть $k = \lceil \log_2 p \rceil$ -- битовая длина числа $p$. Алгоритм Гельфонда имеет  экспоненциальную сложность (число двоичных операций)
    \[ O(\sqrt{p}) = O(e^{\frac{1}{2} \frac{1}{\log_2 e} k}). \]

Наилучшие из известных алгоритмов решения задачи дискретного логарифмирования имеют экспоненциальную сложность порядка
    \[ O(e^{\sqrt{k}}). \]


\subsection[Российский стандарт ЭП ГОСТ Р 34.10-2001]{Российский стандарт ЭП \protect\\ ГОСТ Р 34.10-2001}
\selectlanguage{russian}

Пусть имеются две стороны $A$ и $B$ и канал связи между ними. Сторона $A$ желает передать сообщение $M$ стороне $B$ и подписать его. Сторона $B$ должна проверить правильность подписи, то есть аутентифицировать сторону $A$.

$A$ формирует открытый ключ следующим образом:

\begin{enumerate}
    \item Выбирает простое\index{число!простое} число $p > 2^{255}$.
    \item Записывает уравнение эллиптической кривой:
        \[ E: ~ y^2 = x^3 + a x + b \mod p, \]
        которое определяет группу точек эллиптической кривой $\E(\Z_p)$.
        Выбирает группу, задавая либо случайные числа $0 < a, b < p-1$, либо инвариант $J(E)$:
        \[ J(E) = 1728 \frac{4 a^3}{4 a^3 + 27 b^2} \mod p. \]
        Если кривая задаётся инвариантом $J(E) \in \Z_p$, то он выбирается случайно в интервале $0 < J(E) < 1728$. Для нахождения $a,b$ вычисляется
        \[ K = \frac{J(E)}{1728 - J(E)}, \]
        \[ \begin{array}{l}
            a = 3 K \mod p, \\
            b = 2 K \mod p. \\
        \end{array} \]
    \item Пусть $m$ -- порядок группы точек эллиптической кривой $\E(\Z_p)$. ~Пользователь $A$ подбирает число $n$ и простое\index{число!простое} число $q$ такие, что
        \[ m = n q, ~ 2^{254} < q < 2^{256}, ~ n \geq 1, \]
        где $q$ -- делитель порядка группы.

        В циклической подгруппе порядка $q$ выбирается точка
        \[ P \in \E(\Z_p): ~ q P \equiv 0. \]
    \item Случайно выбирает число $d$ и вычисляет точку $Q = d P$.
    \item Формирует закрытый и открытый ключи:
        \[ \SK = (d), ~ \PK = (p, E, q, P, Q). \]
\end{enumerate}

Теперь сторона $A$ создаёт свою цифровую подпись $S(M)$ сообщения $M$, выполняя следующие действия:
\begin{enumerate}
    \item Вычисляет число $\alpha = h(M)$, где $h$~--- криптографическая хэш-функция, определённая стандартом ГОСТ Р 34.11-94. В российском стандарте длина $h(M)$ равна 256 бит.
    \item Вычисляет $e = \alpha \mod q$.
    \item Случайно выбирает число $k$ и вычисляет точку
        \[ C = k P = (x_c, y_c). \]
    \item Вычисляет $r = x_c \mod q$.
	Если $r = 0$, то выбирает другое $k$.
    \item Вычисляет $s = k e + r d \mod q$.
    \item Формирует подпись
        \[ S(M) = (r, s). \]
\end{enumerate}
Сторона $A$ передаёт стороне $B$ сообщение с подписью
    \[ (M, ~ S(M)). \]

Сторона $B$ проверяет подпись $(r,s)$, выполняя процедуру проверки подписи:
\begin{enumerate}
    \item Вычисляет $\alpha = h(M)$ и $e = \alpha \mod q$. Если $e = 0$, то определяет $e = 1$.
    \item Вычисляет $e^{-1} \mod q$.
    \item Проверяет условия $r < q, ~ r < s$. Если эти условия не выполняются, то подпись отвергается. Если условия выполняются, то процедура продолжается.
    \item Вычисляет числа:
        \[ \begin{array}{l}
            a = s e^{-1} \mod q, \\
            b = -r e^{-1} \mod q. \\
        \end{array} \]
    \item Вычисляет точку:
        \[ \tilde{C} = a P + b Q = (\tilde{x}_c, \tilde{y}_c). \]
        Если подпись верна, должны получить исходную точку $C$.
    \item Проверяет условие $\tilde{x}_{c} \mod q = r$. Если условие выполняется, то подпись принимается, в противном случае -- отвергается.
\end{enumerate}

Рассмотрим вычислительную сложность вскрытия подписи. Предположим, что криптоаналитик ставит своей задачей определение закрытого ключа $d$. Как известно, эта задача является трудной. Для подтверждения этого можно привести такой факт. Был поставлен следующий эксперимент: было выбрано число $p = 2^{97}$ и 1200 персональных компьютеров, которые работали над этой задачей в 16 странах мира, используя процессоры с тактовой частотой 200 МГц. Задача была решена за 53 дня круглосуточной работы. Если взять $p = 2^{256}$, то на решение такой задачи при наличии одного компьютера с частотой процессора 2 ГГц потребуется $10^{22}$ лет.


\section{Длины ключей}
\selectlanguage{russian}

В табл.~\ref{tab:recommended-key-lengths} приведены битовые длины ключей для криптосистем.
%Традиционные рекомендации основаны на аппроксимации существующих алгоритмов для взлома на 10-30 лет вперед.

\begin{table}[!ht]
    \centering
    \caption{Минимальные длины ключей в битах по стандартам России и США\label{tab:recommended-key-lengths}}
    \resizebox{\textwidth}{!}{ \begin{tabular}{|l|c|c|c|c|}
        \hline
        & \multirow{2}{*}{\parbox{1.5cm}{Блочные шифры, $K$}} & \multicolumn{3}{|c|}{Схема ЭП} \\
        \cline{3-5}
        & & \parbox{1.5cm}{RSA\index{криптосистема!RSA}, $n$} & \parbox{2.3cm}{Эллипт. кривые, порядок точки} & \parbox{3.5cm}{Эль-Гамаль\index{криптосистема!Эль-Гамаля} $\mod p$: модуль / порядок (под)группы} \\
        \hline \hline
        \multicolumn{5}{|c|}{Взломано} \\
        \hline
        Биты & 56 & 663 & 109 & 503  \\
        Конкурс & \textsc{DesChal} & RSA-200 & ECC2K-108 &  \\
        Год & 1997 & 2005 & 2000 &  \\
        \hline \hline
        \multicolumn{5}{|c|}{Стандарт России} \\
        \hline
        Биты & 256 &  & 255 & \\
        ГОСТ & 28147—89 & --- & 34.10-2001 & --- \\
        Год & 1989 & & 2001 & \\
%       \hline
%       \multicolumn{2}{|l|}{\parbox{4cm}{Россия: нелицензируемая деятельность}} & \multicolumn{4}{c|}{40} \\
        \hline \hline
        \multicolumn{5}{|c|}{Стандарт США} \\
        \hline
        Биты & 128-256 & 1024-3072 & 151-480 & 1024-3072/160-256 \\
        FIPS № & 197 & draft 186-3 & draft 186-3 & draft 186-3 \\
        Год & 2001 & 2006 & 2006 & 2006 \\
%       \hline
%       \multicolumn{2}{|l|}{\parbox{4cm}{США: экспортные ограничения до 2001 г.}} & 56 & 512 & 112 & 512/112 \\
%       \hline \hline
%       \multicolumn{2}{|l|}{Традиционные} & 80 & 1024 & 160 & 1024/160 \\
%       \cline{3-6}
%       \multicolumn{2}{|l|}{рекомендации} & 112 & 2048 & 224 & 2048/224 \\
%       \hline
%       \multicolumn{2}{|l|}{\parbox{4cm}{Рекомендация Lenstra, Verheul для 2010 г.}} & 78 & 1369 & 146-160 & 1369/138 \\
        \hline
    \end{tabular} }
\end{table}
%}\end{center}


\subsection*{Скорость вычисления ЭП}

Сравним производительность схем ЭП, чтобы продемонстрировать преимущества ЭП вида Эль-Гамаля\index{криптосистема!Эль-Гамаля} перед RSA\index{криптосистема!RSA} для больших ключей. В приложении показано, что в модульной арифметике по модулю числа $n$ с битовой длиной $k \simeq \log_2 n$ операции имеют битовую сложность:
\[ \begin{array}{lcl}
    a^b \mod n & - & O(k^3), \\
    ab \mod n, ~ a^{-1} \mod n & - & O(k^2), \\
    a+b \mod n & - & O(k). \\
\end{array} \]

Так как все описанные схемы ЭП используют возведение в степень по модулю, то битовая сложность -- $O(k^3)$. Оценки количества целочисленных $t$-разрядных умножений при вычислении ЭП имеют вид:
\begin{enumerate}
    \item RSA\index{электронная подпись!RSA}:
        \[ (2 \log_2 n) \cdot \left( \frac{\log_2 n}{t} \right)^2; \]
    \item DSA\index{электронная подпись!DSA} (Digital Signature Algorithm, стандарт США~\cite{FIPS-PUB-186-4}), вычисляемая по принципу Эль-Гамаля\index{криптосистема!Эль-Гамаля} по модулю $p$ и с порядком циклической подгруппы $q$:
        \[ (2 \log_2 q) \cdot \left( \frac{\log_2 p}{t} \right)^2; \]
    \item ГОСТ Р 34.10-2001\index{электронная подпись!ГОСТ Р 34.10-2001} (стандарт России~\cite{GOST-2001}) и ECDSA\index{электронная подпись!ECDSA} (Elliptic Curve Digital Signature Algorithm, стандарт США~\cite{FIPS-PUB-186-4}), вычисляемые по принципу Эль-Гамаля\index{криптосистема!Эль-Гамаля} в группе точек эллиптической кривой по модулю $p$:
        \[ (2 \log_2 p) \cdot 4 \cdot \left( \frac{\log_2 p}{t} \right)^2. \]
\end{enumerate}

В табл.~\ref{tab:signature-rate} приведены оценки скорости вычисления ЭП (оценки числа умножений 64-битовых слов).

\begin{table}[!ht]
    \centering
    \caption{Оценочное число 64-битовых умножений для вычисления ЭП\label{tab:signature-rate}}
    \begin{tabular}{|c|l|c|}
        \hline
        ЭП & Оценочное число 64-битовых умножений \\
        \hline \hline
        RSA\index{электронная подпись!RSA} 1024 & $(2 \cdot 1024) \cdot \left( \frac{1024}{64} \right)^2 \approx$ 500 000 \\
        RSA\index{электронная подпись!RSA} 2048 & 4 000 000 \\
        RSA\index{электронная подпись!RSA} 3072 & 14 000 000 \\
        RSA\index{электронная подпись!RSA} 4096 & 34 000 000 \\
        \hline \hline
        DSA\index{электронная подпись!DSA} 1024/160 & $(2 \cdot 160) \cdot \left( \frac{1024}{64} \right)^2 \approx$ 82 000 \\
        DSA\index{электронная подпись!DSA} 3072/256 & 1 200 000 \\
        \hline \hline
        ECDSA\index{электронная подпись!ECDSA} 160 & $(2 \cdot 160) \cdot 4 \cdot \left( \frac{160}{64} \right)^2 \approx$ 8 000 \\
        ECDSA\index{электронная подпись!ECDSA} 512 & 260 000 \\
        \hline \hline
        ГОСТ Р 34.10-2001\index{электронная подпись!ГОСТ Р 34.10-2001} & $(2 \cdot 256) \cdot 4 \cdot \left( \frac{256}{64} \right)^2 \approx$ 33 000 \\
        \hline
    \end{tabular}
\end{table}

