\subsection{Протокол Нидхема~---~Шрёдера с доверенным центром}\index{протокол!Нидхема~---~Шрёдера}
\selectlanguage{russian}

Рассмотрим ситуацию, когда в сети имеется некоторый надежный (доверенный) сервер (центр) $T$, которому доверяют все пользователи сети. Сервер для работы с абонентами сети использует некоторую систему шифрования $E_S(*)$, где ключ $S=K_{AT}$  известен только $A$ и $T$, но неизвестен остальным участникам сети, $S = K_{BT}$ известен только $B$ и  $T$. Предполагаем, что сервер имеет хороший генератор случайных чисел. Сеансовый ключ сервер вырабатывает по запросу. Стороны $A$ и $B$ могут выбирать разные одноразовые метки.

Приведём в качестве примера упрощенную версию известного \textbf{протокола Нидхема~---~Шрёдера} (Needham~---~Schroeder) с симметричным шифром.
\begin{enumerate}
    \item Сторона $A$ передаёт серверу $T$ реквизиты сторон $A$ и $B$  и некую одноразовую метку $N_A$, которая может быть, например, меткой времени или случайным (одноразовым) числом, что оговаривается заранее:
            \[ A \rightarrow T: ~ A, B, N_A. \]
    \item Сервер $T$ вырабатывает секретный сеансовый ключ $K$ для $A$ и $B$ и отправляет стороне $A$ шифрованное сообщение:
            \[ A \leftarrow T: ~ E_{K_{AT}}(N_A, B, K, E_{K_{BT}}(K, A)). \]
    \item Сторона $A$ расшифровывает сообщение
            \[ D_{K_{AT}}( E_{K_{AT}}(N_A, B, K, E_{K_{BT}}(K, A))) = (N_A, B, K, E_{K_{BT}}(K, A)) \]
        и, чтобы доставить ключ, передаёт стороне $B$ сообщение:
            \[ A \rightarrow B: ~ E_{K_{BT}}(K, A). \]
    \item Сторона $B$ расшифровывает полученное сообщение
            \[ D_{K_{BT}}( E_{K_{BT}}( K,A)) = (K,A) \]
        и получает ключ и реквизиты $A$, которые требуются для того, чтобы сторона $B$ знала, кому отвечать. Кроме того, сторона $B$ дополнительно желает идентифицировать сторону $A$. Для этого $B$ пересылает $A$ зашифрованную одноразовую метку:
            \[ A \leftarrow B: ~ E_{K}(N_B). \]
    \item Сторона $A$ расшифровывает
            \[ D_K( E_K( N_B)) = N_B \]
        и возвращает $B$ изменённую одноразовую метку
            \[ A \rightarrow B: ~ E_K(N_B + 1). \]
    \item Сторона $B$ расшифровывает
            \[ D_K( E_K( N_B + 1)) = N_B + 1, \]
        проверяет $N_B$ и убеждается, что имеет дело со стороной $A$.
    \item Если требуется двусторонняя аутентификация, то аналогично поступают со стороной $A$: на некотором этапе вносится одноразовая метка $N_A$.
\end{enumerate}
