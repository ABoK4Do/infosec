\section{Трёхпроходные протоколы}
\selectlanguage{russian}

Если между Алисой и Бобом существует канал связи, недоступный для модификации злоумышленником (то есть когда применима модель только пассивного криптоаналитика), то даже без предварительного обмена секретными ключами или другой информацией можно воспользоваться идеями, использованными ранее в криптографии на открытых ключах. После описания RSA в 1978 году, в 1980 Ади Шамир предложил использовать криптосистемы, основанные на коммутативных операциях, для передачи информации без предварительного обмена секретными ключами. Если предположить, что передаваемой информацией является выработанный одной из сторон секретный сеансовый ключ, то в общем виде мы получаем следующий трёхпроходной протокол.

Предварительно:

\begin{itemize}
	\item Алиса и Боб соединены незащищённым каналом связи, открытым для прослушивания (но не для модификации) злоумышленником.
	\item Каждая из сторон имеет пару из открытого и закрытого ключей $K_A$, $k_A$, $K_B$, $k_B$ соответственно.
	\item Сторонами выбрана и используется коммутативная функция шифрования:
	\begin{align*}
		D_{A} \left( E_{A} \left( X \right) \right)	&= X		& ~~\forall ~ X, \left\{ K_A, k_a \right\}; \\
		E_{A} \left( E_{B} \left( X \right) \right)	&= E_B \left( E_A \left( X \right) \right) & ~~\forall ~ K_A, K_B, X.
	\end{align*}
\end{itemize}

Протокол состоит из трёх шагов (отсюда и название).
\begin{enumerate}
    \item Алиса создаёт новый секретный сеансовый ключ $K_S$, шифрует его с помощью своего ключа $K_A$ и посылает сообщение Бобу:
        \[ Alice \rightarrow ~ E_A \left( K_S \right) ~ \rightarrow Bob. \]
    \item Боб получает это сообщение, шифрует его с помощью своего ключа $K_B$ и посылает сообщение Алисе:
        \[ Bob \rightarrow ~ E_B \left( E_A \left( K_S \right) \right) ~ \rightarrow Alice. \]
    Алиса, получив сообщение $E_B \left( E_A \left( K_S \right) \right)$, использует свой закрытый ключ $k_A$ для расшифрования:
	\[ D_A \left( E_B \left( E_A \left( K_S \right) \right) \right) = D_A \left( E_A \left( E_B \left( K_S \right) \right) \right) = E_B \left( K_S \right). \]
    \item Алиса передаёт Бобу сообщение, в котором новый секретный сеансовый ключ зашифрован уже только ключом Боба:
        \[ Alice \rightarrow ~ E_B \left( K_S \right) ~ \rightarrow Bob. \]
    Боб, получив сообщение $E_B \left( K_S \right)$, использует свой ключ $k_B$ для расшифрования:
	\[ D_B \left( E_B \left( K_S \right) \right) = K_S. \]
\end{enumerate}

В результате стороны получают общий секретный ключ $K_S$.

Общим недостатком всех подобных протоколов является отсутствие аутентификации сторон. Конечно, в случае пассивного криптоаналитика это не требуется, но в реальной жизни всё-таки нужно рассматривать все возможные модели (в том числе активного криптоаналитика) и использовать такие протоколы, которые предполагают взаимную аутентификацию сторон.

\subsection{Тривиальный вариант}

Приведём пример протокола на основе функции XOR (побитовое сложение 2). Хотя данная функция может использоваться как фундамент для построения систем совершенной криптостойкости (см. главу~\ref{chapter:perfect_secure_systems}), для трёхпроходного протокола это неудачный выбор.

Перед началом протокола обе стороны имеют свои секретные ключи $K_A$ и $K_B$, представляющие собой случайные двоичные последовательности с равномерным распределением символов. Функция шифрования определяется как $E_i( X ) = X \oplus K_i$, где $X$ это сообщение, а $K_i$ -- секретный ключ. Очевидно, что:
\[ \forall i, j, X: E_i \left( E_j \left( X \right) \right) = X \oplus K_j \oplus K_i = X \oplus K_i \oplus K_j = E_j \left( E_i \left( X \right) \right) \]

\begin{enumerate}
    \item Алиса генерирует новый сеансовый ключ, шифрует его своим секретным ключом $K_A$ и посылает Бобу:
            \[\begin{array}{l}
		E_A(K) = K \oplus K_A, \\
		Alice \rightarrow ~ E_A(K) ~ \rightarrow Bob.
	    \end{array}\]
    \item Боб шифрует полученное сообщение уже своим ключом и отправляет Алисе:
            \[\begin{array}{l}
		E_B(E_A(K)) = K \oplus K_A \oplus K_B, \\
		Bob \rightarrow ~ E_B(E_A(K)) ~ \rightarrow Alice.
	    \end{array}\]
    \item Используя коммутативность операции шифрования, Алиса <<снимает>> шифрование своим ключом и отправляет результат Бобу:
            \[ D_A \left( E_B \left( E_A \left( K \right) \right) \right) = K \oplus K_A \oplus K_B \oplus K_A = K \oplus K_B = E_B \left( K \right). \]
            \[ Alice \rightarrow ~ E_B \left( K \right) ~ \rightarrow Bob. \]
    Боб, получив сообщение $K \oplus K_B$, выполняет расшифрование:
            \[ D_B( E_B( K ) ) = K \oplus K_B \oplus K_B = K. \]
    В результате Алиса и Боб знают общий сеансовый ключ $K$.
\end{enumerate}

Предложенный выбор коммутативной функции шифрования совершенной секретности был назван неудачным, так как существуют ситуации, при которых криптоаналитик может определить ключ $K$. Предположим, что криптоаналитик перехватил все три сообщения:
    \[ K \oplus K_A, ~~ K \oplus K_A \oplus K_B, ~~ K \oplus K_B. \]
Сложение по модулю 2 всех трёх сообщений даёт ключ $K$. Поэтому такая система шифрования не применяется.

Теперь приведём протокол надёжной передачи секретного ключа, основанный на экспоненциальной (коммутативной) функции шифрования. Стойкость этого протокола связана с трудностью задачи вычисления дискретного логарифма: при известных значениях $y, g, p$, найти $x$ из уравнения $y = g^x \mod p$.

\subsection{Бесключевой протокол Шамира}\index{протокол!Шамира бесключевой|(}

Стороны предварительно договариваются о большом простом числе $p \sim 2^{1024}$. Каждая из сторон выбирает себе по секретному ключу $a$ и $b$. Эти ключи меньше и взаимно просты с $p-1$. Также стороны приготовили по специальному числу $a'$ и $b'$, которые позволяют им расшифровать сообщение, зашифрованное своим ключом:
\[\begin{array}{l}
a' = a{-1} \mod (p-1), \\
a \times a' = 1 \mod (p-1), \\
\forall X: (X^a)^{a'} = X. \\
\end{array}\]

Последнее выражение верно по следствию из малой теоремы Ферма\index{теорема!Ферма малая}. Операции шифрования и расшифрования определяются следующим образом (на примере Алисы):
\[\begin{array}{ll}
\forall X < p:	& E_A( X ) = X^{a} \mod p, \\
		& D_A( X ) = X^{a'} \mod p, \\
		& D_A( E_A( X ) ) = X^{aa'} = X \mod p. \\
\end{array}\]

\begin{enumerate}
    \item Алиса создаёт новый сеансовый ключ $K < p$, шифрует его своим секретным ключом $a$ и посылает сообщение Бобу:
            \[\begin{array}{l}
		E_A(K) = K ^ a \mod p, \\
		Alice \rightarrow ~ E_A(K) ~ \rightarrow Bob.
	    \end{array}\]
    \item Боб шифрует полученное сообщение уже своим ключом и отправляет Алисе:
            \[\begin{array}{l}
		E_B(E_A(K)) = K^{ab} \mod p, \\
		Bob \rightarrow ~ E_B(E_A(K)) ~ \rightarrow Alice.
	    \end{array}\]
    \item Используя коммутативность операции шифрования, Алиса <<снимает>> шифрование своим ключом и отправляет результат Бобу:
            \[ D_A \left( E_B \left( E_A \left( K \right) \right) \right) = \left( K^{ab} \right) ^{a'} = K^{aa'b} = K^{b} = E_B \left( K \right) \mod p. \]
            \[ Alice \rightarrow ~ E_B \left( K \right) ~ \rightarrow Bob. \]
    Боб, получив сообщение $K \oplus K_B$, выполняет расшифрование:
            \[ D_B( E_B( K ) ) = K \oplus K_B \oplus K_B = K. \]
    В результате Алиса и Боб знают общий сеансовый ключ $K$.
\end{enumerate}

Предположим, что криптоаналитик перехватил три сообщения:
\[ \begin{array}{l}
    y_1 = K^a \mod p, \\
    y_2 = K^{ab} \mod p, \\
    y_3 = K^b \mod p. \\
\end{array} \]

Чтобы найти ключ $K$, криптоаналитику надо решить систему из этих трёх уравнений, что имеет очень большую вычислительную сложность, неприемлемую с практической точки зрения, если все три числа $a, b, ab$ достаточно велики. Предположим, что $a$ (или $b$) мало. Тогда, вычисляя последовательные степени $y_3$ (или $y_1$), можно найти $a$ (или $b$), сравнивая результат с $y_2$. Зная $a$, легко найти $a^{-1}\mod(p-1)$ и $K=(y_1)^{a^{-1}}\mod p$.

\index{протокол!Шамира бесключевой|)}

\subsection{Криптосистема Мэсси~---~Омуры}\index{протокол!Мэсси~---~Омуры|(}\index{криптосистема!Мэсси~---~Омуры|(}

В 1982 году Джеймс Мэсси и Джим Омура заявили патент (\langen{James Massey, Jim K. Omura},~\cite{Massey:Omura:1986}), улучшающий (по их мнению) бесключевой протокол Шамира. В качестве операции шифрования вместо возведения в степень в мультипликативной группе $\Z_p^*$ они предложили использовать возведение в степень в поле Галуа $\GF{2^n}$. Секретный ключ каждой стороны (для Алисы -- $a$) должен удовлетворять условиям:
\[ \begin{array}{l}
 a \in \GF{2^n}, \\
 gfd \left( a, x^{ n-1 } + x^{ n-2 } + ... + x + 1 \right) = 1. \\
\end{array} \]

В остальном протокол выглядит аналогично.

\index{протокол!Мэсси~---~Омуры|)}\index{криптосистема!Мэсси~---~Омуры|)}