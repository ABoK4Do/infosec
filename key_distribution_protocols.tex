\chapter{Распространение ключей}\index{протокол!распространения ключей}\label{chapter-key-distribution-protocols}
\selectlanguage{russian}

Задачей распространения ключей между двумя пользователями является создание секретных псевдослучайных сеансовых ключей шифрования и аутентификации сообщений. Пользователи предварительно создают и обмениваются ключами аутентификации один раз. В дальнейшем для создания защищённой связи пользователи производят взаимную аутентификацию и вырабатывают сеансовые ключи\index{ключ!сеансовый}.

\section[Трехэтапный протокол Шамира]{Трехэтапный протокол Шамира на коммутативных шифрах}
\selectlanguage{russian}

Предположим, что две стороны $A$ и $B$ соединены незащищённым каналом связи. Каждая из этих сторон имеет свой секретный ключ: $A$ имеет ключ $K_A$, $B$ имеет ключ $K_B$. Сторона $A$ должна создать общий секретный ключ $K$ и передать стороне $B$.

Для решения этой задачи используют трехэтапный протокол Шамира с тремя <<замками>>. \textbf{Протокол Шамира}\index{протокол!Шамира} построен на \emph{коммутативных} функциях шифрования, для которых выполняется условие:
    \[ E_{K_{B}} (E_{K_{A}}(K))=E_{K_{A}} (E_{K_{B}}(K)). \]

Протокол предполагает следующие процедуры.
\begin{enumerate}
    \item $A$ создаёт секретный ключ $K$, шифрует его своей системой шифрования с помощью своего ключа $K_A$ и посылает сообщение стороне $B$:
        \[ A \rightarrow B: ~ E_{K_A}(K). \]
    \item $B$ получает это сообщение, шифрует его с помощью своего ключа $K_B$ и посылает сообщение стороне $A$:
        \[ A \leftarrow B: ~ E_{K_B}( E_{K_A}( K)). \]
    \item Сторона $A$, получив сообщение $E_{K_B}(E_{K_A}(K))$, использует свой секретный ключ $K_A$ для расшифрования:
            \[ D_{K_A}(E_{K_B} (E_{K_A}(K))) = E_{K_B}(K). \]
        Сторона $A$ передаёт стороне $B$ сообщение:
        \[ A \rightarrow B: ~ E_{K_B}(K). \]
    \item Сторона $B$, получив сообщение $E_{K_B}(K)$, использует свой секретный ключ $K_B$ для расшифрования:
            \[ D_{K_B}(E_{K_B}(K)) = K. \]
        В результате стороны получают общий секретный ключ $K$.
\end{enumerate}

Приведём пример неудачного шифрования с использованием коммутативных функций.

\begin{enumerate}
    \item $A$ имеет функцию шифрования совершенной секретности $E_{K_A}(K) = K \oplus K_A$, где $K_A$ -- двоичная последовательность с равномерным распределением символов. $A$ посылает это сообщение стороне $B$:
            \[ A \rightarrow B: ~ E_{K_A}(K) = K \oplus K_A. \]
    \item $B$ использует такую же функцию шифрования совершенной секретности с ключом $K_B$ (двоичная последовательность с равномерным распределением символов). $B$ шифрует полученное сообщение и отправляет $A$:
            \[ A \leftarrow B: ~ E_{K_A}(K) \oplus K_B = K \oplus K_A \oplus K_B. \]
    \item Сторона $A$, получив сообщение $K \oplus K_A \oplus K_B$, выполняет расшифрование:
            \[ K \oplus K_A \oplus K_B \oplus K_A = K \oplus K_B. \]
        Сторона $A$ передаёт стороне $B$ сообщение:
            \[ A \rightarrow B: ~ K \oplus K_B. \]
    \item Сторона $B$, получив сообщение $K \oplus K_B$, выполняет расшифрование:
            \[ K \oplus K_B \oplus K_B = K. \]
        Обе стороны получают общий секретный ключ $K$.
\end{enumerate}

Предложенный выбор коммутативной функции шифрования совершенной секретности был назван неудачным, так как существуют ситуации, при которых криптоаналитик может определить ключ $K$. Предположим, что криптоаналитик перехватил все три сообщения:
    \[ K \oplus K_A, ~~ K \oplus K_A \oplus K_B, ~~ K \oplus K_B. \]
Сложение по модулю 2 всех трех сообщений даёт ключ $K$. Поэтому такая система шифрования не применяется.

Теперь приведём протокол надежной передачи секретного ключа, основанный на экспоненциальной (коммутативной) функции шифрования. Стойкость этого протокола связана с трудной задачей -- задачей вычисления дискретного логарифма: известны значения $y, g, p$, найти $x$ в уравнении $y = g^x \mod p$.

\textbf{Протокол Шамира распространения ключей}
Выбирают большое простое\index{число!простое} число $p\sim 2^{1024}$ и используют его как открытый ключ.

\begin{enumerate}
    \item Сторона $A$ задаёт общий секретный ключ $K <p$ и выбирает целое число $a$, взаимно простое с $p-1$. $A$ вычисляет и посылает сообщение стороне $B$:
            \[ A \rightarrow B: ~ K^a \mod p. \]
        Существует число $c$ такое, что $a c =1 \mod (p-1)$, то есть $a c = 1 + l (p-1)$, где $l$ -- целое число. Число $c$ будет использовано стороной $A$ на следующем этапе.
    \item Сторона $B$ выбирает целое число $b$, взаимно простое с $p-1$, используя полученное сообщение, вычисляет и посылает сообщение стороне $A$:
            \[ A \leftarrow B: ~ (K^a)^b \mod p. \]
    \item Сторона $A$, получив сообщение, вычисляет
        \[ \left( K^{ab} \right)^c = K^{1 + l (p-1) b} = K^b \cdot K^{l (p-1) b} = K^b \mod p. \]
        Здесь применена малая теорема Ферма\index{теорема!Ферма малая}: $K^{p-1} = 1 \mod p$, поэтому $\left( K^{p-1} \right)^{lb} = 1 \mod p$.
        $A$ посылает $B$ сообщение:
            \[ A \rightarrow B: ~ K^b \mod p. \]
    \item Сторона $B$, получив сообщение $K^{b}\mod p$, вычисляет
        \[ (K^b \mod p)^d = K^{bd} \mod p = K. \]
\end{enumerate}

Теперь проверим криптостойкость этого протокола. Предположим, что криптоаналитик перехватил три сообщения:
\[ \begin{array}{l}
    y_1 = K^a \mod p, \\
    y_2 = K^{ab} \mod p, \\
    y_3 = K^b \mod p. \\
\end{array} \]
Чтобы найти ключ $K$, надо решить систему из этих трех уравнений, что имеет очень большую вычислительную сложность, неприемлемую с практической точки зрения, если все три числа $a, b, ab$ достаточно велики. Предположим, что $a$ (или $b$) мало. Тогда, вычисляя последовательные степени $y_3$ (или $y_1$), можно найти  $a$ (или $b$), сравнивая результат с $y_2$. Зная $a$, легко найти $a^{-1}\mod(p-1)$ и $K=(y_1)^{a^{-1}}\mod p$.

Недостатком этого протокола является отсутствие аутентификации сторон. Следовательно, нужно дополнительно использовать цифровую подпись при передаче сообщения.


\section{Симметричные протоколы}

\subsection{Аутентификация и атаки воспроизведения}

Рассмотрим такую ситуацию: обе стороны $A$ и $B$ имеют общий долговременный ключ $K_{AB}$ и симметричную систему шифрования. Нужно выработать сеансовый секретный ключ $K$. Сторона $A$ создаёт ключ $K$ и желает его передать стороне $B$.

\begin{enumerate}
    \item Для этого сторона $A$ с помощью общего ключа $K_{AB}$ создаёт и передаёт $B$ шифрованное сообщение:
            \[ A \rightarrow B: ~ E_{K_{AB}}(K, B, A). \]
        В этом сообщении имеются так называемые поля -- $(B,A)$ -- информация для дополнительного подтверждения.
    \item Сторона $B$, используя общий ключ $K_{AB}$, расшифровывает полученное сообщение:
            \[ D_{K_{AB}}( E_{K_{AB}}( K, B, A)) = (K, B, A). \]
        В результате сторона $B$ получает сеансовый ключ $K$ и дополнительные данные $(B,A)$.
\end{enumerate}

Недостаток этого протокола состоит в том, что криптоаналитик может перехватывать сообщения и через некоторое время пересылать их стороне $A$.

Рассмотрим другие варианты решения задачи о передаче сеансового ключа.
Задача остаётся прежней: обе стороны $A$ и $B$ имеют общий долговременный секретный ключ $K_{AB}$, сторона $A$ должна выработать сеансовый секретный ключ $K$ и доставить его стороне $B$.

Протокол включает \emph{метки времени} -- информацию о моменте $t_A$ отправки сообщения и моменте получения сообщения $t_B$.

\begin{enumerate}
    \item Сторона $A$ вырабатывает $K$, с помощью долговременного ключа $K_{AB}$ создаёт шифрованное сообщение с меткой времени $t_A$ и передаёт его стороне $B$:
            \[ A \rightarrow B: ~ E_{K_{AB}}(K, t_A). \]
    \item Сторона $B$ получает сообщение и расшифровывает его с помощью общего ключа:
            \[ D_{K_{AB}}( E_{K_{AB}}( K, t_A)) = (K, t_A). \]
        В результате $B$ получает $(K, t_A)$, то есть секретный ключ и метку времени. $B$ измеряет время прихода $t_B$ и интервал запаздывания. Если $|t_B - t_A| \leq \delta$, то $B$ аутентифицирует $A$.
\end{enumerate}
Метка времени является одноразовой меткой и защищает от атак воспроизведения ранее записанных сообщений.

Рассмотрим другой способ передачи ключа с дополнительной информацией в виде \emph{одноразовых случайных меток} (nonce -- number used once) вместо меток времени. Протокол передачи состоит в следующем.

\begin{enumerate}
    \item Сторона $A$ вырабатывает случайное число $r_A$, шифрует сообщение, в котором $(r_A, A)$ -- реквизиты $A$, и передаёт его стороне $B$:
            \[ A \rightarrow B: ~ E_{K_{AB}}(r_A, A). \]
    \item Сторона $B$ вырабатывает сеансовый ключ $K$, создаёт шифрованное сообщение и посылает его $A$:
            \[ A \leftarrow B: ~ E_{K_{AB}}(K, r_A, A). \]
    \item Сторона $A$ расшифровывает полученное сообщение:
            \[ D_{K_{AB}}( E_{K_{AB}}( K, r_A, A)) = (K, r_A, A). \]
        В результате $A$ получает сеансовый ключ и подтверждение своих реквизитов, что является дополнительной аутентификацией.
\end{enumerate}

Предположим, что сторона $B$ тоже желает убедиться, что имеет дело со стороной $A$. Тогда этот протокол следует дополнить передачей реквизитов $B$. По-прежнему считаем, что у $A$ и $B$ общая система шифрования с долговременным секретным ключом $K_{AB}$.

\begin{enumerate}
    \item Сторона $A$ вырабатывает случайное число $r_A$, шифрует и передаёт стороне $B$ сообщение, в котором $(r_A, A)$ -- реквизиты $A$:
            \[ A \rightarrow B: ~ E_{K_{AB}}(r_A, A). \]
    \item Сторона $B$ вырабатывает случайное число $r_B$ и отправляет стороне $A$ зашифрованное сообщение:
            \[ A \leftarrow B: ~ E_{K_{AB}}(K_B, r_B, r_A, A), \]
        где $K_B$ -- ключ $B$.
     \item Сторона $A$ осуществляет расшифрование:
            \[ D_{K_{AB}}(E_{K_{AB}}(K_B, r_B, r_A, A)) = (K_B, r_B, r_A, A), \]
        и получает ключ $K_B$ и реквизиты $r_B, r_A, A$. Для аутентификации себя сторона $A$ создаёт свой ключ $K_A$ и отправляет стороне $B$ шифрованное сообщение:
            \[ A \rightarrow B: ~ E_{K_{AB}}(K_A, r_B, r_A, B). \]
     \item Сторона $B$ осуществляет расшифрование:
            \[ D_{K_{AB}}(E_{K_{AB}}(K_A, r_B, r_A, B)) = (K_A, r_B, r_A, B), \]
        которое определяет ключ $K_A$ и аутентифицирует $A$.
\end{enumerate}

Таким образом, обе стороны имеют в своём распоряжении ключи $K_A, K_B$ в качестве сеансовых секретных ключей.

\subsection{Протокол с ключевым кодом аутентификации}

При использовании хэш-функции $K = h(K_{A} ~\|~ K_{B})$ происходит усиление секретности. Здесь $(K_{A} ~\|~ K_{B})$ -- конкатенация $K_{A} $ и $K_{B}$.

% Достоинства: предположим, $K_{A} ,K_{B} $ -- не обладают «хорошими» свойствами случайности (биты распределены неравномерно или зависимы друг от друга), то есть, $P_{K_{A} ,K_{B} } (0)=\frac{1}{2} -\varepsilon $, где $\varepsilon $ - мало, но не 0. Тогда вероятность того, что этот бит в \emph{K }будет равным нулю, $P_{K} (0)=\frac{1}{2} -\varepsilon ',\varepsilon '<\varepsilon $- усиление секретности.

Вычисление хэш-значения, как правило, выполняется быстрее, чем расшифрование. Поэтому были разработаны протоколы, в которых вместо функции шифрования используется имитовставка\index{имитовставка} на основе хэш-функции ($\MAC_K$). Рассмотрим протокол такого рода.
\begin{enumerate}
    \item Сторона $A$ вырабатывает сеансовый ключ $K$, создаёт сообщение, используя одноразовую метку $t_{A}$, и пересылает его стороне $B$:
            \[ A \rightarrow B: ~ t_A, ~ B, ~ K \oplus \MAC_{K_{AB}}( t_A, B), ~ \MAC_{K_{AB}}(K, t_A, B). \]
    \item Сторона $B$ вычисляет
            \[ \MAC_{K_{AB}}(t_A, B) \oplus K \oplus \MAC_{K_{AB}}(t_A, B) = K \]
        и получает сеансовый ключ $K$.
\end{enumerate}

Заметим, что криптоаналитик может добавить в поле случайную последовательность, тогда вместо $K$ получаем <<$K$ плюс помеха>>. Вмешательство криптоаналитика будет выявлено благодаря наличию четвёртого поля в сообщении. Используя полученное значение $K$, вычисляют $\MAC_{K_{AB}}(K, t_A, B)$ и сравнивают с четвёртым полем. Если значения совпадают, то вмешательства криптоаналитика не было.

\subsection{Протокол Нидхема~---~Шрёдера с доверенным центром}\index{протокол!Нидхема~---~Шрёдера}
\selectlanguage{russian}

Рассмотрим ситуацию, когда в сети имеется некоторый надежный (доверенный) сервер (центр) $T$, которому доверяют все пользователи сети. Сервер для работы с абонентами сети использует некоторую систему шифрования $E_S(*)$, где ключ $S=K_{AT}$  известен только $A$ и $T$, но неизвестен остальным участникам сети, $S = K_{BT}$ известен только $B$ и  $T$. Предполагаем, что сервер имеет хороший генератор случайных чисел. Сеансовый ключ сервер вырабатывает по запросу. Стороны $A$ и $B$ могут выбирать разные одноразовые метки.

Приведем в качестве примера упрощенную версию известного \textbf{протокола Нидхема~---~Шрёдера} (Needham~---~Schroeder) с симметричным шифром.
\begin{enumerate}
    \item Сторона $A$ передаёт серверу $T$ реквизиты сторон $A$ и $B$  и некую одноразовую метку $N_A$, которая может быть, например, меткой времени или случайным (одноразовым) числом, что оговаривается заранее:
            \[ A \rightarrow T: ~ A, B, N_A. \]
    \item Сервер $T$ вырабатывает секретный сеансовый ключ $K$ для $A$ и $B$ и отправляет стороне $A$ шифрованное сообщение:
            \[ A \leftarrow T: ~ E_{K_{AT}}(N_A, B, K, E_{K_{BT}}(K, A)). \]
    \item Сторона $A$ расшифровывает сообщение
            \[ D_{K_{AT}}( E_{K_{AT}}(N_A, B, K, E_{K_{BT}}(K, A))) = (N_A, B, K, E_{K_{BT}}(K, A)) \]
        и, чтобы доставить ключ, передаёт стороне $B$ сообщение:
            \[ A \rightarrow B: ~ E_{K_{BT}}(K, A). \]
    \item Сторона $B$ расшифровывает полученное сообщение
            \[ D_{K_{BT}}( E_{K_{BT}}( K,A)) = (K,A) \]
        и получает ключ и реквизиты $A$, которые требуются для того, чтобы сторона $B$ знала, кому отвечать. Кроме того, сторона $B$ дополнительно желает идентифицировать сторону $A$. Для этого $B$ пересылает $A$ зашифрованную одноразовую метку:
            \[ A \leftarrow B: ~ E_{K}(N_B). \]
    \item Сторона $A$ расшифровывает
            \[ D_K( E_K( N_B)) = N_B \]
        и возвращает $B$ изменённую одноразовую метку
            \[ A \rightarrow B: ~ E_K(N_B + 1). \]
    \item Сторона $B$ расшифровывает
            \[ D_K( E_K( N_B + 1)) = N_B + 1, \]
        проверяет $N_B$ и убеждается, что имеет дело со стороной $A$.
    \item Если требуется двусторонняя аутентификация, то аналогично поступают со стороной $A$: на некотором этапе вносится одноразовая метка $N_A$.
\end{enumerate}


\section{Асимметричные протоколы}

Асимметричные протоколы, или же протоколы, основанные на криптосистемах с открытыми ключами, позволяют ослабить требования к предварительному этапу протоколов. Вместо общего секретного ключа, который должны иметь две стороны (либо обе стороны и доверенный центр), в рассматриваемых ниже протоколах стороны должны предварительно обменяться открытыми ключами (между собой либо между собой и доверенным центром). Такой предварительный обмен может проходить по открытому каналу связи, в предположении, что криптоаналитик не может повлиять на содержимое канала связи на данном этапе.

\subsection{Простой протокол}

Рассмотрим протокол распространения ключей с помощью асимметричных шифров. Введём обозначения: $K_B$ -- открытый ключ стороны $B$, а $K_A$ -- открытый ключ стороны $A$. Протокол включает три сеанса обмена информацией.
\begin{enumerate}
    \item В первом сеансе сторона $A$ посылает стороне $B$ сообщение:
            \[ A \rightarrow B: ~ E_{K_B}(K_1, A), \]
        где $K_1$ -- ключ, выработанный стороной $A$.
    \item Сторона $B$ получает $(K_1, A)$ и передаёт стороне $A$ наряду с другой информацией свой ключ $K_2$ в сообщении, зашифрованном с помощью открытого ключа $K_A$:
            \[ A \leftarrow B: ~ E_{K_A}(K_2, K_1, B). \]
    \item Сторона $A$ получает и расшифровывает сообщение $(K_2, K_1, B)$. Во время третьего сеанса сторона $A$, чтобы подтвердить, что она знает ключ $K_2$, посылает стороне $B$ сообщение:
            \[ A \rightarrow B: ~ E_{K_B}(K_2). \]
\end{enumerate}
Общий ключ формируется из двух ключей: $K_1$ и $K_2$.

\subsection{Протоколы с цифровыми подписями}

Существуют протоколы обмена, в которых перед началом обмена ключами генерируются подписи сторон $A$ и $B$, соответственно $S_A(m)$ и $S_B(m)$. В этих протоколах можно использовать различные одноразовые метки. Рассмотрим пример.
\begin{enumerate}
    \item Сторона $A$ выбирает ключ $K$ и вырабатывает сообщение:
            \[ \left( K, ~ t_A, ~ S_A(K, t_A, B) \right), \]
        где $t_A$ -- метка времени. Зашифрованное сообщение передаёт стороне $B$:
        \[ A \rightarrow B: ~ E_{K_B}(K, ~ t_A, ~ S_A(K, t_A, B)). \]
    \item Сторона $B$ получает это сообщение, расшифровывает $\left( K, ~ t_A, ~ S_A(K, t_A, B) \right)$ и вырабатывает свою метку времени $t_B$. Проверка считается успешной, если $|t_B - t_A | < \delta $. Сторона $B$ знает свои реквизиты и может осуществлять проверку подписи.
\end{enumerate}

Имеется второй вариант протокола, в котором шифрование и подпись выполняются раздельно.
\begin{enumerate}
    \item Сторона $A$ вырабатывает ключ $K$, использует одноразовую метку (или метку времени) $t_{A}$ и передаёт стороне $B$ два различных зашифрованных сообщения:
            \[ \begin{array}{ll}
                A \rightarrow B: & ~ E_{K_B}(K, t_A), \\
                A \rightarrow B: & ~ S_A(K, t_A, B). \\
            \end{array} \]
    \item Сторона $B$ получает это сообщение, расшифровывает $K, t_A$ и, добавив свои реквизиты $B$, может проверить подпись $S_A(K, t_A, B)$.
\end{enumerate}

В третьем варианте протокола сначала производится шифрование, потом подпись.
\begin{enumerate}
    \item Сторона $A$ вырабатывает ключ $K$, использует одноразовую случайную метку или метку времени $t_A$ и передаёт стороне $B$ сообщение:
        \[ A \rightarrow B: ~ t_A, ~ E_{K_B}(K, A), ~ S_A(t_A, ~ K, ~ E_{K_B}(K, A)). \]
    \item Сторона $B$ получает это сообщение, расшифровывает $\left( t_A, ~ K, ~ A, ~ E_{K_B}(K, A) \right)$ и проверяет подпись $S_A(t_A, ~ K, ~ E_{K_B}(K, A))$.
\end{enumerate}

\subsection{Протокол Диффи~---~Хеллмана}\index{протокол!Диффи~---~Хеллмана}
\selectlanguage{russian}

Алгоритм с открытым ключом впервые был предложен У.~Диффи (W.~Diffie) и М.~Хеллманом (M.E.~Hellman) в работе 1976 года <<Новые направления в криптографии>>(<<New directions in cryptography>>,~\cite{Diffie:Hellman:1976}).

Рассмотрим протокол Диффи~---~Хеллмана обмена информацией двух сторон $A$ и $B$. Задача состоит в том, чтобы создать общий сеансовый ключ.

Пусть $p$ -- большое простое число\index{число!простое}, $g$ -- примитивный элемент группы $\Z_p^*$, ~ $y = g^x \mod p$, причем $p,y,g$ известны заранее. Функцию $y=g^{x} \mod p$ считаем однонаправленной, т.~е. вычисление функции при известном значении аргумента является легкой задачей, а ее обращение (нахождение аргумента) при известном значении функции -- трудной.

Протокол обмена состоит из следующих действий.
\begin{enumerate}
    \item Сторона $A$ выбирает случайное число $x, ~ 2 \leq x \leq (p-1)$, вычисляет и передаёт стороне $B$ сообщение:
        \[ A \rightarrow B: ~ g^x \mod p. \]
    \item Сторона $B$ выбирает случайное число $y, ~ 2\leq y \leq (p-1)$, вычисляет и передаёт стороне $A$:
        \[ A \leftarrow B: ~ g^y \mod p. \]
    \item Сторона $A$, используя известные ей значения $x,g^{y} \mod p$, вычисляет ключ
        \[ K_{A} =(g^{y})^{x}\mod p=g^{xy} \mod p. \]
    \item Сторона $B$, используя известные ей значения $y,g^{x} \mod p$, вычисляет ключ:
        \[ K_{B} =(g^{x})^{y}\mod p=g^{xy}\mod p. \]
        В результате получаем равенство $K_A = K_B = K$.
\end{enumerate}

Таким способом создан общий секретный сеансовый ключ. В каждом новом сеансе используется этот же протокол для создания нового сеансового ключа.

Рассмотрим протокол Диффи~---~Хеллмана в ситуации, когда имеются три легальных пользователя $A,B,C$.

Каждая из сторон $A,B,C$ вырабатывает случайные числа $x,y,z$ соответственно и держит их в секрете.

\begin{enumerate}
    \item Первый этап обмена информацией аналогичен вышеописанному обмену информацией между двумя сторонами:
        \begin{enumerate}
            \item $A \rightarrow B: ~ g^x \mod p$.
            \item $B \rightarrow C: ~ g^y \mod p$.
            \item $C \rightarrow A: ~ g^z \mod p$.
        \end{enumerate}
    \item Второй этап состоит из передач сообщений:
        \begin{enumerate}
            \item $A \rightarrow B: ~ (g^z)^x = g^{zx} \mod p$.
            \item $B \rightarrow C: ~ (g^x)^y = g^{xy} \mod p$.
            \item $C \rightarrow A: ~ (g^y)^z = g^{yz} \mod p$.
        \end{enumerate}
    \item На завершающем третьем этапе стороны вычисляют:
        \begin{enumerate}
            \item $A: ~ K_A = (g^{yz})^x = g^{xyz} \mod p$.
            \item $B: ~ K_A = (g^{zx})^y = g^{xyz} \mod p$.
            \item $C: ~ K_A = (g^{xy})^z = g^{xyz} \mod p$.
        \end{enumerate}
\end{enumerate}

Как видно из произведенных действий, выработанные сторонами $A, B, C$ ключи совпадают: $K_A = K_B = K_C = K$. Следовательно, создан общий секретный сеансовый ключ $K$ для трёх участников.

Таким же образом можно построить протокол Диффи~---~Хеллмана для любого числа легальных пользователей.

Рассмотрим этот двусторонний протокол с точки зрения криптоаналитика, желающего узнать ключ $K$. Предположим, ему удалось перехватить сообщения $g^{x}\mod p$ и $g^{y}\mod p $. Используя заранее известные данные $g,p $ и эти сообщения, криптоаналитик старается найти хотя бы одно из чисел $(x,y)$, то есть решить задачу дискретного логарифма. В настоящее время эта задача считается вычислительно трудной при обычно выбираемых значениях $p\sim 2^{1024}$.

Существует атака активного криптоаналитика\index{криптоаналитик!активный}, названная <<человек посредине>> (man-in-the-middle)\index{атака!<<человек посередине>>}. Пусть имеются две легальные стороны $A$ и $B$ и нелегальная сторона $E$, активный криптоаналитик\index{криптоаналитик!активный}, который имеет возможность перехватывать и подменять сообщения как от $A$, так и от $B$:
    \[ A \leftrightsquigarrow E \leftrightsquigarrow B. \]
    %\[ A \leftrightarrow E \leftrightarrow B. \]

\begin{enumerate}
    \item Подмена ключей.
        \begin{enumerate}
            \item Сторона $A$ передаёт стороне $B$ сообщение:
                \[ A \overset{E}{\nrightarrow} B: ~ g^x \mod p. \]
            \item Сторона $E$ перехватывает сообщение $g^x \mod p$, сохраняет его и, зная $g$, передаёт стороне $B$ свое сообщение:
                \[ E \rightarrow B: ~ g^z \mod p. \]
            \item Сторона $B$ передаёт стороне $A$ сообщение:
                \[ A \overset{E}{\nleftarrow} B: ~ g^y \mod p. \]
            \item Сторона $E$ перехватывает сообщение $g^y \mod p$, сохраняет его и передаёт стороне $A$ свое сообщение
                \[ A \leftarrow E: ~ g^z \mod p \]
                или какое-то другое.
            \item Таким образом, между сторонами $A$ и $E$ образуется общий секретный ключ $K_{AE}$, между $B$ и $E$ -- ключ $K_{BE}$, причем $A$ и $B$ не знают, что у них ключи со стороной $E$, а не с друг другом
                \[ \begin{array} {l}
                    K_{AE} = g^{xz} \mod p, \\
                    K_{BE} = g^{yz} \mod p. \\
                \end{array} \]

        \end{enumerate}
    \item Подмена сообщений.
        \begin{enumerate}
            \item Сторона $A$ посылает $B$ сообщение $m$, зашифрованное на ключе $K_{AE}$:
                % \rightsquigarrow
                \[ A \overset{E}{\nrightarrow} B: ~ E_{K_{AE}}(m). \]
            \item Сторона $E$ перехватывает сообщение, расшифровывает с ключом $K_{AE}$, возможно, подменяет на $m'$, зашифровывает с ключом $K_{BE}$ и посылает $B$:
                \[ E \rightarrow B: ~ E_{K_{BE}}(m'). \]
            \item То же самое происходит при обратной передаче от $B$ к $A$.
        \end{enumerate}
\end{enumerate}

Криптоаналитик $E$ имеет возможность перехватывать и подменять все передаваемые сообщения. Если по тексту письма нельзя обнаружить участие криптоаналитика в обмене информацией, то атака <<человек посередине>>\index{атака!<<человек посередине>>} успешна.

Существует несколько протоколов для преодоления атаки этого типа.


%\section{Протоколы с аутентификацией}

\subsection{Односторонняя аутентификация}

\textbf{Протокол Эль-Гамаля}\index{протокол!Эль-Гамаля} относится к протоколам с аутентификацией одного из двух легальных пользователей.
\selectlanguage{russian}
\begin{enumerate}
    \item Для начала стороны выбирают общие параметры $p, g$, где $p$ -- большое простое число, где $g$ -- примитивный элемент поля $\Z_p^*$.
    \item Сторона $B$ создает свои секретный и открытый ключи:
            \[ \SK_B = b, ~ \PK_B = g^b \mod p, \]
        $b$ -- случайное секретное число, $2 \leq b \leq p-1$.

        Открытый ключ $\PK_B$ находится в общем открытом доступе для всех сторон, поэтому криптоаналитик $E$ не может подменить его -- подмена будет заметна.
    \item Сторона $A$ вырабатывает свой секрет $x$, сеансовый ключ
            \[ K_A = (\PK_B)^x = g^{bx} \mod p \]
        и отправляет $B$:
            \[ A \rightarrow B: ~ g^x \mod p. \]
    \item Сторона $B$, получив от $A$ число $g^x \mod p$, использует его и свой секрет $\SK_B = b$, чтобы создать свой ключ
            \[ K_B = (g^x)^{\SK_B} = g^{bx} \mod p, \]
        то есть сеансовые ключи обеих сторон совпадают:
            \[ K_A = K_B = K. \]
\end{enumerate}

Достоинством этого протокола является следующее его свойство. Если ключи $K_A$ и $K_B$ совпали и стороны могут обмениваться информацией, то сторона $A$ аутентифицирует сторону $B$, так как для шифрования она использовала открытый ключ $B$, который не может быть незметно подменен и только сторона $B$ может расшифровывать сообщения.

Что касается криптоаналитика в качестве <<человека-посередине>>, то он может отправлять ложные сообщения, но не может узнать ключ $K$ и читать сообщения.

Есть протоколы, в которых стороны, осуществляющие обмен информацией, являются равноправными. Они называются протоколами взаимной аутентификации.


\subsection{Взаимная аутентификация шифрованием}
\selectlanguage{russian}

К протоколам взаимной аутентификации принадлежит семейство протоколов, разработанных Ц.~Мацумото (\langen{Tsutomu Matsumoto}), И.~Такашима (\langen{Youichi Takashima}) и Х.~Имаи (\langen{Hideki Imai}) и названных по первым буквам фамилий авторов -- \emph{протоколы MTI}\index{протоколы!MTI}.

Здесь к открытым данным относятся:
    \[ p, ~~ g, ~~ \PK_A = g^a \mod p, ~~ \PK_B = g^b \mod p. \]
Каждый пользователь $A$ и $B$ обладает парой долговременных ключей для \emph{схемы шифрования с открытым ключом}: закрытым ключом расшифрования $\SK$ и открытым ключом шифрования $\PK$.
\[ \begin{array}{ll}
    A: & ~ \SK_A = a, ~~ \PK_A = g^a \mod p, \\
    B: & ~ \SK_B = b, ~~ \PK_B = g^b \mod p. \\
\end{array} \]

\textbf{Протокол MTI}:
\begin{enumerate}
    \item Сторона $A$ генерирует случайное число $x, ~ 2\leq x\leq p-1$, создаёт и отправляет $B$ сообщение:
        \[ A \rightarrow B: ~ g^x \mod p. \]
    \item Сторона $B$ генерирует случайное число $y, ~ 2\leq y\leq p-1$, создаёт и отправляет $A$ сообщение:
        \[ A \leftarrow B: ~ g^y \mod p. \]
    \item Сторона $A$, используя открытые данные и полученное сообщение, создаёт сеансовый ключ:
        \[ K_A = (g^b)^x \cdot (g^y)^a = g^{bx+ay} \mod p. \]
    \item Сторона $B$, используя открытые данные и полученное сообщение, создаёт сеансовый ключ:
        \[ K_B = (g^x)^b \cdot (g^a)^y = g^{bx+ay} \mod p. \]
        Сеансовые ключи обеих сторон совпадают:
        \[ K_{A} =K_{B} = K. \]
\end{enumerate}

В описанном протоколе, как и в протоколе Эль-Гамаля\index{криптосистема!Эль-Гамаля}, происходит взаимная аутентификация сторон: открытые ключи сторон незаметно подменить невозможно. Наблюдая сообщения протокола, вычислить $g^{bx+ay}$ можно, только если известны значения $a,x$ или $b,y$, что представляет собой задачу дискретного логарифмирования, вычислительно трудную на сегодняшний день.


\subsection{Взаимная аутентификации схемой ЭП}
\selectlanguage{russian}

\textbf{Протокол STS (Station-To-Station)}\index{протокол!Station-To-Station} предназначен для систем мобильной связи. Он использует идеи протокола Диффи~---~Хеллмана\index{протокол!Диффи~---~Хеллмана} и идеи системы RSA\index{криптосистема!RSA}.

Здесь открытые общедоступные данные
    \[ p, ~ g, ~ \PK_A, ~ \PK_B. \]
Каждая из сторон $A$ и $B$ обладает долговременной парой ключей: секретным ключом подписания $\SK$ и открытым ключом проверки подписи $\PK$ для \emph{схемы ЭП}.
\[ \begin{array}{ll}
    A: & ~ \SK_A, ~~ \PK_A, \\
    B: & ~ \SK_B, ~~ \PK_B. \\
\end{array} \]
Подписи к сообщению $m$ сторон $A$ и $B$ имеют вид:
\[ \begin{array}{ll}
    A: & ~ S_A(m) = \textrm{ЭП}_{\SK_A}(H(m)), \\
    B: & ~ S_B(m) = \textrm{ЭП}_{\SK_B}(H(m)), \\
\end{array} \]
$H(m)$ -- криптографическая хэш-функция от сообщения $m$.

Протокол состоит из трех раундов обмена информацией между сторонами $A$ и $B$.
\begin{enumerate}
    \item Сторона $A$ создает секретное случайное число $2 \leq x \leq p-1$ и отправляет $B$:
            \[ A \rightarrow B: ~ g^x \mod p. \]
    \item Сторона $B$ создает секретное случайное число $2 \leq y \leq p-1$, вычисляет общий секретный ключ
            \[ K = (g^x)^y = g^{xy} \mod p, \]
        с помощью которого создает шифрованное сообщение $E_K(S_B(g^x, g^y))$ для аутентификации, и отправляет $A$:
            \[ A \leftarrow B: ~ \left( g^y \mod p, ~~ E_K( S_B( g^x, g^y)) \right). \]
    \item Сторона $A$ с помощью $x, g^y \mod p$ вычисляет общий секретный ключ
            \[ K = (g^y)^x \mod p = g^{xy} \mod p \]
        и расшифровывает сообщение
            \[ D_K( E_K( S_B( g^x, g^y))) = S_B( g^x, g^y). \]
            Затем аутентифицирует сторону $B$, проверяя подпись $S_B$ открытым ключом $\PK_B$. Вычисляет и пересылает стороне $B$ сообщение:
            \[ A \rightarrow B: ~ E_K( S_A( g^x, g^y)). \]
    \item Сторона $B$ расшифровывает принятое сообщение
            \[ D_K( E_K( S_A( g^x, g^y))) = S_A( g^x, g^y) \]
        и осуществляет аутентификацию, выполняя проверку подписи $S_A$ с помощью открытого ключа $\PK_A$.
\end{enumerate}


\subsection{Взаимная аутентификация с доверенным центром}
\selectlanguage{russian}

В  \textbf{протоколе Жиро}\index{протокол!Жиро} (Marc Girault,~\cite{Girault:1990, Girault:1991}) участвуют три стороны -- $A$, $B$ и надёжный центр $T$.
\begin{enumerate}
    \item У стороны $T$ есть открытый и закрытый ключи криптосистемы RSA\index{криптосистема!RSA},
        \[ n=pq, ~ e, ~ d=e^{-1} \mod \varphi(n) \]
        с дополнительным параметром $g$ -- генератором подгруппы максимально возможного порядка мультипликативной группы $\Z_n^*$:
        \[ \begin{array}{l}
            \PK_T = (n, e, g) ~~\text{открытый ключ}, \\
            \SK_T = (d) ~~ \text{закрытый ключ}. \\
        \end{array} \]
    \item Стороны $A$ и $B$ независимо создают свои открытые и закрытые ключи, обмениваясь информацией с центром $T$ по надёжному защищённому каналу. Стороны $A$ и $B$ выбирают свои закрытые ключи
        \[ \begin{array}{l}
            \SK_A = a, \\
            \SK_B = b, \\
        \end{array} \]
     и отправляют центру сообщения:
        \[ \begin{array}{ll}
            A \rightarrow T: & ~ I_A, ~~ g^{-\SK_A} = g^{-a} \mod n, \\
            B \rightarrow T: & ~ I_B, ~~ g^{-\SK_B} = g^{-b} \mod n, \\
        \end{array} \]
        где $I_A, I_B$ -- числовые идентификаторы сторон.
    \item Центр  $T$ вычисляет открытые ключи для $A$ и $B$ и также по надёжному каналу передаёт им:
        \[ \begin{array}{ll}
            A \leftarrow T: & ~ \PK_A = (g^{-\SK_A} - I_A)^{\SK_T} = (g^{-a} - I_A)^d \mod n, \\
            B \leftarrow T: & ~ \PK_B = (g^{-\SK_B} - I_B)^{\SK_T} = (g^{-b} - I_B)^d \mod n. \\
        \end{array} \]
    \item Теперь стороны $A$ и $B$ могут создать общий секретный симметричный сеансовый ключ. Например $A$ находит:
        \[ \begin{array}{ll}
            A: ~~ K_A & = ~ (\PK_B^e + I_B)^{\SK_A} ~ = \\
                & = ~ (((g^{-b} - I_B)^d)^e + I_B)^a ~ = \\
                & = ~ (g^{-b} - I_B + I_B)^a ~ = \\
                & = ~ g^{-ab} \mod n. \\
        \end{array} \]
        Аналогично $B$ вычисляет
            \[ K_B = (\PK_A^e + I_A)^{\SK_B} = g^{-ab} \mod n. \]
        Как видно, ключи одинаковы:
            \[ K = K_A = K_B = g^{-ab} \mod n. \]
\end{enumerate}


\subsection{Схема Блома распределения парных ключей}
\selectlanguage{russian}

Рассмотрим распределение ключей по \textbf{схеме Блома}\index{распределение секрета!Блома} (Blom), в которой каждые два пользователя из общего числа $N$ пользователей имеют доступ к общему секретному ключу, ключи различных пар различны.

По-прежнему,
    \[ \{ U_1, U_2, \dots, U_N \} \]
легальные пользователи, $\Z_p$ -- кольцо целых чисел.

Построим симметричный многочлен
    \[ f(x,y) = \sum_{i=1}^k \sum_{j=1}^k a_{ij} x^i y^j, \]
    \[ a_{ij} \in \Z_p, ~ a_{ij} = a_{ji}. \]

Возьмем набор чисел $r_1, r_2, \dots, r_N$, где $r_i$ -- открытый ключ пользователя $U_i$, ~ $r_i \in \Z_p$.

Каждый пользователь $U_i$ получает многочлен $f(x,y)$ и вместо $y$ подставляет свое значение $r_i$, так что получается $N$ многочленов $f(x, r_i), ~ i = 1, 2, \dots, N$.

Каждые два участника коалиции должны иметь общий ключ. Пусть, например, $U_1$ и $U_2$ хотят создать общий ключ. Тогда пользователь $U_1$, используя $f(x, r_1)$ и зная $r_2$, вычисляет
    \[ K_{12} = f(r_2, r_1). \]

Пользователь $U_2$, используя $f(x, r_2)$ и зная $r_1$, вычисляет
    \[ K_{1,2} = f(r_1, r_2). \]

Так как для выбранного многочлена справедливо равенство
    \[ f(r_1, r_2) = f(r_2, r_1), \]
то
    \[ K_{12} = K_{21}. \]
Таким образом, два участника коалиции создали общий ключ. Таким же образом поступают и другие пары пользователей. Третий пользователь, не участник коалиции, не может подобрать ключ, так как это представляет собой трудную задачу в вычислительном смысле.

%\section{Пример предварительного распределения ключей и разделения секрета в схеме Блома в виде многочленов}
\example
В схеме распределения ключей Блома для $N=4$ пользователей доверенный Центр выбирает:
\begin{enumerate}
    \item модуль $p = 17$ поля $\GF{p}$;
    \item свой секретный симметричный многочлен от двух переменных
        \[ f(x,y) = a + b (x + y) + c x y \mod p \]
        над полем $\GF{p}$;
    \item открытые ключи для каждого пользователя
        \[ r_1 = 5, ~ r_2 = 9, ~ r_3 = 14, ~ r_4 = 3; \]
    \item вычисляет и секретно раздает многочлен $S_i(x)$ каждому пользователю $U_i$:
        \[ \begin{array}{l}
            S_1(x) = f(x, r_1) = 1 + 2x \mod p, \\
            S_2(x) = f(x, r_2) = 3 + 10x \mod p, \\
            S_3(x) = f(x, r_3) = 14 + 3x \mod p, \\
            S_4(x) = f(x, r_4) = 0 + 15x \mod p. \\
        \end{array} \]
\end{enumerate}
Найдем ключи и восстановим секретный многочлен доверенного Центра.

Секретные сеансовые ключи пользователей равны
    \[ K_{ij} = K_{ji} = S_i(r_j) = S_j(r_i): \]
\[ \begin{array}{lcl}
    K_{1,2} = K_{2,1} = 2, & & K_{1,3} = K_{3,1} = 12, \\
    K_{1,4} = K_{4,1} = 7, & & K_{2,3} = K_{3,2} = 7, \\
    K_{2,4} = K_{4,2} = 16, & & K_{3,4} = K_{4,3} = 6. \\
\end{array} \]

По любым 3 многочленам пользователей можно восстановить секретный многочлен Центра. Коэффициенты секретного многочлена Центра равны $a=7, b=9, c=2$.
\exampleend

%\section{Схема предварительного распределения ключей и разделения секрета Блома}
%
%Центр выбирает секретную симметрическую $(k \times k)$-матрицу $F$ над полем $\GF{p}$, где $p$ -- простое. Каждому пользователю $i$ Центр создает и выдает открытый ключ $P_i$, который является $k$-мерным вектором над $\GF{p}$, и секретный ключ $S_i = F \cdot P_i$, $k$-мерный вектор.
%
%Когда два пользователя $i$ и $j$ хотят создать секретный сессионный ключ для обмена сообщениями, они обмениваются открытыми ключами $P_i, P_j$ и вычисляют секретный сессионный ключ $K_{ij} = S_i P_j^T = S_j P_i^T$.
%
%%Если известны открытые ключи $k$ пользователей, то...
%%TODO
%
%%Схема Блома используется в High-bandwidth Digital Content Protection (HDCP), разрботанной Intel для применения в DVD плеерах и телевидении высокой четкости.


В этом разделе были рассмотрены протоколы, в которых ключи вырабатываются в процессе обмена информацией.
%Существует и другой подход, который будет рассмотрен в следующих разделах.
