\subsection{Взаимная аутентификация шифрованием}
\selectlanguage{russian}

К протоколам взаимной аутентификации принадлежит семейство протоколов, разработанных  Т. Мацумото (T. Matsumoto), И. Такашима (Y. Takashita) и Х. Имаи (H. Imai) и названных по первым буквам фамилий авторов -- \textbf{протокол MTI}\index{протокол!MTI}.

Здесь к открытым данным относятся
    \[ p, ~~ g, ~~ \PK_A = g^a \mod p, ~~ \PK_B = g^b \mod p. \]
Каждый пользователь $A$ и $B$ обладает парой  долговременных ключей для \emph{схемы шифрования с открытым ключом}: секретный ключ расшифрования $\SK$ и открытый ключ шифрования  $\PK$.
\[ \begin{array}{ll}
    A: & ~ \SK_A = a, ~~ \PK_A = g^a \mod p, \\
    B: & ~ \SK_B = b, ~~ \PK_B = g^b \mod p. \\
\end{array} \]

\textbf{протокол MTI}
\begin{enumerate}
    \item Сторона $A$ генерирует случайное число $x, ~ 2\leq x\leq p-1$, создает и отправляет $B$ сообщение:
        \[ A \rightarrow B: ~ g^x \mod p. \]
    \item Сторона $B$ генерирует случайное число $y, ~ 2\leq y\leq p-1$, создает и отправляет $A$ сообщение:
        \[ A \leftarrow B: ~ g^y \mod p. \]
    \item Сторона $A$, используя открытые данные и полученное сообщение, создает  сеансовый ключ:
        \[ K_A = (g^b)^x \cdot (g^y)^a = g^{bx+ay} \mod p. \]
    \item Сторона $B$, используя открытые данные и полученное сообщение, создает  сеансовый ключ:
        \[ K_B = (g^x)^b \cdot (g^a)^y = g^{bx+ay} \mod p. \]
        Сеансовые ключи обоих сторон совпадают:
        \[ K_{A} =K_{B} = K. \]
\end{enumerate}

В описанном протоколе происходит взаимная аутентификация сторон как и в протоколе Эль-Гамаля: открытые ключи сторон незаметно подменить невозможно. Наблюдая сообщения протокола, вычислить $g^{bx+ay}$ можно, только если известны значения $a,x$ или $b,y$, что представляет собой задачу дискретного логарифма, трудную в вычислительном смысле в настоящее время.
