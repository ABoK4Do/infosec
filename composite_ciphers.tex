\subsubsection{Композиционные шифры}
\selectlanguage{russian}

Почти все современные шифры являются \emph{композиционными}~\cite{AlZKCh:2001}. В них применяются несколько различных методов шифрования к одному и тому же открытому тексту. Другое их название -- \emph{составные шифры}. Впервые понятие <<составные шифры>> было введено в работе Клода Шеннона (\langen{Claude Elwood Shannon}).

В современных криптосистемах шифры замены и перестановок используются многократно, образуя составные (композиционные) шифры.

