\subsection{Криптосистемы с секретным ключом}
\selectlanguage{russian}

Криптосистема  называется криптосистемой \textbf{с секретным ключом}\index{криптосистема!с секретным ключом}, если оба ключа $K_1$ и $K_2$ держатся в секрете.
%(см. \refl{pic:Encrypt}).
Криптосистема называется \textbf{симметричной}, когда ключи совпадают
    \[ K_1 = K_2 = K, ~~ \set{K}_E = \set{K}_D. \]
    Или в случае, когда один ключ можно вычислить из другого за полиномиальное время и с полиномиальными ресурсами.
Конкретный симметричный ключ $K$\index{ключ!симметричный}, используемый при шифровании и расшифровании, должен быть известен легальным пользователям $A$ и $B$ и храниться в секрете от нелегального пользователя $E$\index{ключ!секретный}.

Естественными являются следующие требования к функциям шифрования и расшифрования  $E_K(X)$ и $D_K(Y)$:
\begin{itemize}
  \item при заданных парах аргументов $(K,X)$ (соответственно $(K,Y)$) вычисление значений $E_K(X)$ и $D_K(Y)$ должно быть <<легкой>> задачей, то есть требовать небольших вычислительных ресурсов от легальных пользователей $A$ и $B$;
  \item если известен аргумент $Y$ и не известен ключ $K$, то нахождение $X$ обращением  функции $Y = E_K(X)$
      %функций $Y = E_K(X)$ или $X = D_K(Y)$
      должно быть <<трудной>> задачей, то есть за пределом вычислительных возможностей криптоаналитика $E$.
\end{itemize}

Функция $y = f(x)$, для которой нахождение прообраза $x$ по образу $y$ вычислительно невозможно, называется \textbf{однонаправленной}\index{функция!однонаправленная}.

Симметричные криптосистемы делятся на потоковые и блоковые. \textbf{Потоковые}\index{криптосистема!потоковая} криптосистемы осуществляют посимвольное (например, побитовое или побайтовое) шифрование потока символов неограниченной длины. В основном, потоковые криптосистемы реализуются аппаратно и применяются в беспроводных сетях передачи данных из-за необходимости сразу же (т.е. с ограниченной задержкой) передавать данные по мере их поступления.

\textbf{Блоковые}\index{криптосистема!блоковая} криптосистемы шифруют один блок символов, то есть последовательность фиксированной длины. Как правило, размер блока равен 64, 128 или 256 бит. Для шифрования блоковым шифром  сообщения произвольной длины оно разбивается на блоки, которые шифруются последовательно, один за другим. Зашифрованные блоки формируют шифротекст. Шифрование блока может быть зависимым от предыдущего блока -- так называемый \textbf{режим сцепления блоков}.

Применение симметричных криптосистем требует решения  задачи \textbf{распределения секретных ключей} между легальными пользователями -- т.е. надежной генерации и доставки уникальных секретных ключей для всех пар пользователей.
