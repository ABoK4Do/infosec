\subsection{Парадокс дней рождений}
\selectlanguage{russian}

Найдем вероятность $P(n)$ того, что в группе из $n$ человек хотя бы двое имеют день рождения в один день года. Кроме того, найдем минимальный размер группы, в которой дни рождения совпадают хотя бы у двоих с вероятностью не менее $\frac{1}{2}$\index{парадокс дней рождений}.

Вероятность того, что $n$ человек ($n < 365$) не имеют общего дня рождения, есть
\[
    \bar{P}(n) = 1 \cdot \left( 1 - \frac{1}{365} \right) \cdot \left(1 - \frac{2}{365} \right)  \dots  \left( 1 - \frac{n-1}{365} \right) = \prod\limits_{i=0}^{n-1} \left( 1 - \frac{i}{365} \right).
\]
Аппроксимируя $1-x \leq e^{-x}$, находим
    \[ \bar{P}(n) \approx \prod\limits_{i=0}^{n-1} e^{-\frac{i}{365}} = e^{-\frac{n(n-1)}{2} \cdot \frac{1}{365}} \approx e^{-\frac{n^2}{2} \cdot \frac{1}{365}}. \]
Вероятность того, что хотя бы 2 человека из $n$ имеют общий день рождения, есть
    \[ P(n) = 1 - \bar{P}(n) \approx 1 -  e^{-\frac{n^2}{2} \cdot \frac{1}{365}}. \]

Найдем такое число $n_{1/2}$, чтобы выполнялось условие $P(n_{1/2}) \geq \frac{1}{2}$. Подставляя это значение в формулу для вероятности,  получим $\frac{1}{2} \geq e^{-\frac{n_{1/2}^2}{2} \cdot \frac{1}{365}}$. Следовательно,
\[
    n_{1/2} \geq \sqrt{2 \ln 2 \cdot 365} \geq 23.
\]

Парадокс дней рождений заключается в том, что вероятность совпадения дней рождения хотя бы у одной пары в группе, много меньше вероятности совпадения дней рождения двух людей не в группе.

%Средний размер $\bar{n}$ группы, в которой есть хотя бы 2 человека с одним днем рождения, асимптотически $\bar{n} \approx 1 + \sqrt{\frac{\pi}{2}365}$.

Обобщим эту задачу на выборку из $n$ случайных элементов с повторами из множества  $2^k$ различных элементов и получим формулы для вероятности и числа $n$:
\[
    P(n) \approx 1 -  e^{-\frac{n^2}{2} 2^{-k}}, ~~
    n_{1/2} \approx \sqrt{2 \ln 2} \cdot 2^{k/2}, ~~
%    \bar{n} \approx \sqrt{\frac{\pi}{2}} \cdot 2^{k/2}.
\]
