\subsection{Китайская теорема об остатках}
\selectlanguage{russian}

Пусть
    \[ n = \prod\limits_{i=1}^{r} n_i, ~ \gcd(n_i, n_j) = 1. \]

 \textbf{Китайская теорема об остатках}\index{теорема!китайская об остатках}\index{CRT} (Chinese Remainder Theorem, CRT) для взаимно простых $\gcd(n_i,n_j) = 1, ~ i \neq j$, утверждает:
\begin{enumerate}
    \item Между
        \[ a \mod n \]
        и вектором
        \[ (a_1 = a \mod n_1, ~~ a_2 = a \mod n_2,  \dots,  a_r = a \mod n_r) \]
        существует взаимно однозначное соответствие.
    \item Сохраняются операции сложения и умножения:
        \[ (a \pm b \mod n) ~\Leftrightarrow~ (a_1 \pm b_1 \mod n_1,  \dots,  a_r \pm b_r \mod n_r), \]
        \[ (a b \mod n) ~\Leftrightarrow~ (a_1 b_1 \mod n_1,  \dots,  a_r b_r \mod n_r). \]
\end{enumerate}

Китайская теорема означает, что
    \[ a \mod n ~~\equiv~~ (a \mod n_1, ~ a \mod n_2, ~ \dots, ~ a \mod n_r) \]
и все вычисления по $\mod n$ (сложение, умножение, вычисление обратного элемента) эквиваленты вычислениям по $\mod n_i$.

Чтобы от вектора (остатков) перейти к числу, следует произвести следующие вычисления:
        \[ N_i = \frac{n}{n_i}, \]
        \[ M_i = N_i^{-1} \mod n_i, \]
        \[ a = \sum\limits_{i=1}^{r} a_i M_i N_i \mod n. \]
        Для проверки равенства достаточно найти вычет (остаток от деления) последнего уравнения по $\mod n_i$.

        Сложность перехода в векторную форму имеет порядок:
        \[ O(k^2), ~ k = \lceil \log_2 n \rceil. \]
Теорема используется для решения систем линейных модульных уравнений и для ускорения вычислений.

Пусть битовая длина $n$ равна $k$, и пусть все $n_i$ имеют одинаковую битовую длину $k / r$. Тогда операция умножения в векторном виде будет в
    \[ \frac{k^2}{r \left( k \middle/ r \right)^2 } = r \]
раз быстрее.

Операция $c = m^e \mod n$ занимает $O(k^3)$ битовых операций. Если перейти к вычислениям по модулям $n_i$, то возведение в степень можно вычислить в
    \[ \frac{k^3}{r \left( k \middle/ r \right)^3 } = r^2 \]
раз быстрее, коэффициенты результирующего вектора равны
    \[ c_i ~=~ \left( m \mod n_i \right)^{e \mod \varphi(n_i)} \mod n_i, ~ i = 1, \dots, r. \]
Заметим, однако, что для криптографических приложений модули с большим количеством сомножителей применять нельзя из-за катастрофического снижения криптостойкости (снижения сложности разложения модуля на множители).


\subsection[Решение систем линейных уравнений]{Решение систем линейных уравнений}

\example
Решим для примера систему линейных уравнений. Применим CRT и а) для разложения одного уравнения по составному модулю на систему по взаимно простым модулям, и б) для нахождения конечного решения по системе уравнений:
\[
    \begin{cases}
        9 x = 8 \mod 11, \\
        5 x = 7 \mod 12, \\
        x = 5 \mod 6, \\
        122 x = 118 \mod 240; \\
    \end{cases}
    \Rightarrow ~~
    \begin{cases}
        x = 8 \cdot 9^{-1} \mod 11, \\
        x = 7 \cdot 5^{-1} \mod 12, \\
        x = 5 \mod 6, \\
        x = 59 \cdot 61^{-1} \mod 120; \\
    \end{cases}
    \Rightarrow
\] \[
    \Rightarrow ~~
    \begin{cases}
        x = -4 \mod 11, \\
        x = -1 \mod 12, \\
        x = -1 \mod 6, \\
        x = -1 \mod 120; \\
    \end{cases}
    \Rightarrow ~~
    \begin{cases}
        x = -4 \mod 11, \\
        \begin{cases}
            x = -1 \mod 3, \\
            x = -1 \mod 4, \\
        \end{cases} \\
        \begin{cases}
            x = -1 \mod 3, \\
            x = -1 \mod 2, \\
        \end{cases} \\
        \begin{cases}
            x = -1 \mod 8, \\
            x = -1 \mod 3, \\
            x = -1 \mod 5; \\
        \end{cases} \\
    \end{cases}
    \Rightarrow
\] \[
    \Rightarrow ~~
    \begin{cases}
        x = -4 \mod 11, \\
        x = -1 \mod 3, \\
        x = -1 \mod 8, \\
        x = -1 \mod 5. \\
    \end{cases}
\]
Все модули попарно взаимно простые, поэтому применима китайская теорема об остатках:
    \[ n_1 = 11, ~ n_2 = 3, ~ n_3 = 8, ~ n_4 = 5, \]
    \[ n = n_1 n_2 n_3 n_4 = 1320, \]
    \[ N_i = \frac{n}{n_i} ~~ \Rightarrow ~~ N_1 = 120, ~ N_2 = 440, ~ N_3 = 165, ~ N_4 = 264, \]
    \[ M_i = N_i^{-1} \mod n_i, ~~ \Rightarrow ~~ M_1 = 10, ~ M_2 = 2, ~ M_3 = 5, ~ M_4 = 4, \]
    \[ a_1 = -4, ~ a_2 = -1, ~ a_3 = -1, ~ a_4 = -1, \]
    \[ x ~=~ \sum_{i=1}^4 a_i N_i M_i \mod n ~=~ 359 \mod 1320. \]
Ответ: $x ~=~ 359 \mod 1320$.
\exampleend
