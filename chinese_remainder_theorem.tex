\subsection{Китайская теорема об остатках}\label{section-chinese-remainder-theorem}\index{теорема!китайская об остатках}
\selectlanguage{russian}

Китайская теорема об остатках (Chinese Remainder Theorem, CRT), приписываемая китайскому математику Сунь Цзы (пиньинь: S\={u}n Z\v{i}, примерно III~век~до~н.~э.), утверждает о существовании и единственности (с точностью до некоторого модуля) числа $x$, заданного по множеству остатков от деления на попарно взаимно простые числа $n_1, n_2, n_3, \dots, n_k$.

\begin{theorem} Если натуральные числа $n_1, n_2, n_3, \dots, n_k$ попарно взаимно просты, то для любых целых $r_1, r_2, r_3, \dots, r_k$ таких, что $0 \leq r_i < n_i$, найдётся число $x$, которое при делении на $n_i$ даёт остаток $r_i$ для всех $1 \leq i \leq k$. Более того, любые два таких числа $x_1$ и $x_2$, имеющие одинаковые остатки $r_1, r_2, \dots, r_k$, удовлетворяют сравнению:
\[ \begin{array}{l}
	x_1 \equiv x_2 \mod n, \\
	n = n_1 \times n_2 \times \dots \times n_k.
\end{array} \]
\end{theorem}

А формула, приведённая в труде другого китайского математика Циня Цзюшао (пиньинь: Q\'{i}n Ji\v{u}sh\'{a}o, XIII~век~н.~э.), позволяет найти искомое число:

\[ \begin{array}{l}
	x = \sum\limits_{i=1}^k r_i \cdot e_i, \\
	e_i = \frac{n}{n_i} \cdot \left( \left(\frac{n}{n_i}\right)^{-1} \mod n_i \right), i = 1, \dots, k.
\end{array} \]

Китайская теорема об остатках позволяет рассматривать набор попарно взаимно простых чисел $n_1, n_2, n_3, \dots, n_k$ как некоторый <<базис>>, в котором число $0 \leq x < n$ можно задать с помощью <<координат>> $r_1, r_2, r_3, \dots, r_k$. При этом операции сложения и умножения чисел можно выразить через операции сложения и умножения соответствующих остатков:

\[ \begin{array}{l}
	\forall a, b, c, \begin{array}{l}
		a_i = a \mod n_i, \\
		b_i = b \mod n_i, \\
		c_i = c \mod n_i
	\end{array}: \\
	\left( c \equiv a + b \mod n \right) \Leftrightarrow \left( c_i \equiv a_i + b_i \mod n_i, i = 1, \dots, k \right), \\
	\left( c \equiv a \times b \mod n \right) \Leftrightarrow \left( c_i \equiv a_i \times b_i \mod n_i, i = 1, \dots, k \right).
\end{array} \]

Сложность перехода в векторную форму имеет порядок:
        \[ O( \lceil \log_2 n \rceil ^2). \]

Теорема используется для решения систем линейных модульных уравнений и для ускорения вычислений.

Пусть битовая длина $n$ равна $l$, и пусть все $n_i$ имеют одинаковую битовую длину $k / r$. Тогда операция умножения в векторном виде будет в
    \[ \frac{l^2}{r \left( l \middle/ r \right)^2 } = r \]
раз быстрее.

Операция $c = m^e \mod n$ занимает $O(l^3)$ битовых операций. Если перейти к вычислениям по модулям $n_i$, то возведение в степень можно вычислить в
    \[ \frac{l^3}{r \left( l \middle/ r \right)^3 } = r^2 \]
раз быстрее, коэффициенты результирующего вектора равны
    \[ c_i ~=~ \left( m \mod n_i \right)^{e \mod \varphi(n_i)} \mod n_i, ~ i = 1, \dots, k. \]

\subsection{Решение систем линейных уравнений}

\example
Решим для примера систему линейных уравнений. Применим CRT и а) для разложения одного уравнения по составному модулю на систему по взаимно простым модулям, и б) для нахождения конечного решения системы уравнений:
\[
    \begin{cases}
        9 x \equiv 8 \mod 11, \\
        5 x \equiv 7 \mod 12, \\
        x \equiv 5 \mod 6, \\
        122 x \equiv 118 \mod 240; \\
    \end{cases}
    \Rightarrow ~~
    \begin{cases}
        x \equiv 8 \cdot 9^{-1} \mod 11, \\
        x \equiv 7 \cdot 5^{-1} \mod 12, \\
        x \equiv 5 \mod 6, \\
        x \equiv 59 \cdot 61^{-1} \mod 120; \\
    \end{cases}
    \Rightarrow
\] \[
    \Rightarrow ~~
    \begin{cases}
        x \equiv -4 \mod 11, \\
        x \equiv -1 \mod 12, \\
        x \equiv -1 \mod 6, \\
        x \equiv -1 \mod 120; \\
    \end{cases}
    \Rightarrow ~~
    \begin{cases}
        x \equiv -4 \mod 11, \\
        \begin{cases}
            x \equiv -1 \mod 3, \\
            x \equiv -1 \mod 4, \\
        \end{cases} \\
        \begin{cases}
            x \equiv -1 \mod 3, \\
            x \equiv -1 \mod 2, \\
        \end{cases} \\
        \begin{cases}
            x \equiv -1 \mod 8, \\
            x \equiv -1 \mod 3, \\
            x \equiv -1 \mod 5; \\
        \end{cases} \\
    \end{cases}
    \Rightarrow
\] \[
    \Rightarrow ~~
    \begin{cases}
        x \equiv -4 \mod 11, \\
        x \equiv -1 \mod 3, \\
        x \equiv -1 \mod 8, \\
        x \equiv -1 \mod 5. \\
    \end{cases}
\]
Все модули попарно взаимно простые, поэтому применима китайская теорема об остатках\index{теорема!китайская об остатках}:
\[\begin{array}{l}
	n_1 = 11, ~ n_2 = 3, ~ n_3 = 8, ~ n_4 = 5, \\
	n = n_1 \times n_2 \times n_3 \times n_4 = 1320, \\
	n / n_i  = \left\{ 120, 440, 165, 264 \right\}, \\
	\left( n / n_i \right)^{-1} \mod n_i = \left\{ 10, 2, 5, 4 \right\}, \\
	e_i = \left\{ 1200, 880, 825, 1056 \right\}, \\
	x = -4 \cdot 1200 - 1 \cdot 880 - 1 \cdot 825 - 1 \cdot 1056 = -7561, \\
	x \equiv 359 \mod 1320.
\end{array}\]

Аналогично, если переписать последнюю систему с положительными остатками:
\[\begin{array}{l}
	\begin{cases}
		x \equiv 7 \mod 11, \\
		x \equiv 2 \mod 3, \\
		x \equiv 7 \mod 8, \\
		x \equiv 4 \mod 5 \\
	\end{cases}, \\
	x = 7 \cdot 1200 + 2 \cdot 880 + 7 \cdot 825 + 4 \cdot 1056 = 20159, \\
	x \equiv 359 \mod 1320.
\end{array}\]

Ответ: $x \equiv 359 \mod 1320$.
\exampleend
