\section{Полиграммный шифр замены Хилла}\index{шифр!Хилла|(}
\selectlanguage{russian}

Если при шифровании преобразуются более двух букв открытого текста, то шифр называется \emph{полиграммным}\index{шифр!полиграммный}. Первый полиграммный шифр предложил Лестер Хилл в 1929 году (\langen{Lester Sanders Hill}, \cite{Hill:1929, Hill:1931}). Это был первый шифр, который позволял оперировать более чем тремя символами за один такт.

В шифре Хилла текст предварительно преобразуют в цифровую форму и разбивают на последовательности (блоки) по $n$ последовательных цифр. Такие последовательности называются \emph{$n$-граммами}. Выбирают обратимую по модулю $m$  $(n \times n)$-матрицу $\mathbf{A} = (a_{ij})$, где $m$ -- число букв в алфавите. Выбирают случайный $n$-вектор $\mathbf{f} = (f_1, \dots, f_n)$. После чего $n$-грамма открытого текста $\mathbf{x} = (x_1, x_2, \dots, x_n)$ заменяется $n$-граммой шифрованного текста $\mathbf{y} = (y_1, y_2, \dots, y_n)$ по формуле:
    \[ \mathbf{y} = \mathbf{x} \mathbf{A} + \mathbf{f} \mod m. \]
Расшифрование проводится по правилу:
    \[ \mathbf{x} = (\mathbf{y} - \mathbf{f}) \mathbf{A}^{-1} \mod m. \]

\example
Приведём пример шифрования с помощью шифра Хилла. Преобразуем английский алфавит в числовую форму (m = 26) следующим образом:
\[ \text{a} \rightarrow 0, ~ \text{b} \rightarrow 1, ~ \text{c} \rightarrow 2, ~ \dots, ~ \text{z} \rightarrow 25. \]
%\[ \begin{array}{cccccccccccccc}
%    a & b & c & d & e & f & g & h & i & j &  k &  l &  m &  n \\
%    0 & 1 & 2 & 3 & 4 & 5 & 6 & 7 & 8 & 9 & 10 & 11 & 12 & 13
%\end{array} \]
%\[ \begin{array}{cccccccccccc}
%     o &  p &  q &  r &  s &  t &  u &  v &  w &  x &  y &  z  \\
%    14 & 15 & 16 & 17 & 18 & 19 & 20 & 21 & 22 & 23 & 24 & 25
%\end{array} \]

Выберем для примера $n = 2$. Запишем фразу <<Wheatstone was the inventor>> из предыдущего примера (первая строка таблицы). Каждой букве поставим в соответствие её номер в алфавите (вторая строка):
\begin{center} \resizebox{\textwidth}{!}{ \begin{tabular}{|*{12}c|}
    \hline
    w, h & e,a & t,s & t,o & n,e & w,a & s,t & h,e & i,n & v,e & n,t & o,r \\
    \hline
    22, 7 & 4,0 & 19,18 & 19,14 & 13,4 & 22,0 & 18,19 & 7,4 & 8,13 & 21,4 & 13,19 & 14,17 \\
    \hline
\end{tabular} } \end{center}

Выберем матрицу шифрования $A$ в виде:
\[
    \mathbf{A} = \left( \begin{array}{cc}
        5 & 8 \\
        3 & 5 \\
    \end{array} \right).
\]

Эта матрица обратима по $\Mod 26$, так как её определитель равен $1$ и взаимно прост с числом букв английского алфавита $m=26$. Обратная матрица равна:
\[
    \mathbf{A}^{-1} = \left( \begin{array}{cc}
        5  & 18 \\
        23 & 5
    \end{array} \right) \mod 26.
\]

Выберем вектор $\mathbf{f} = (4, 2)$. Первая числовая пара открытого текста $\mathbf{x} = (\text{w}, \text{h}) = (22, 7)$ зашифрована в виде:
\[
    \mathbf{y} = \mathbf{x} \mathbf{A} + \mathbf{f} =
        (22, 7)
        \left( \begin{array}{cc}
            5 & 8 \\
            3 & 5
        \end{array} \right) +
        (4, 2) = (14, 3) \mod 26
\]
или в буквенном виде $(\text{o}, \text{d})$.

Повторяя вычисления для всех пар, получим полный шифрованный текст в числовом виде (третья строка) или в буквенном виде (четвёртая строка):
\begin{center} \resizebox{\textwidth}{!}{ \begin{tabular}{|*{12}c|}
    \hline
    w, h & e, a & t, s & t, o & n, e & w, a & s, t & h, e & i, n & v, e & n, t & o, r \\
    22, 7 & 4, 0 & 19, 18 & 19, 14 & 13, 4 & 22, 0 & 18, 19 & 7, 4 & 8, 13 & 21, 4 & 13, 19 & 14, 17 \\
    \hline
    14, 3 & 24, 22 & 9, 21 & 3, 9 & 23, 1 & 10, 8 & 12, 19 & 19, 23 & 18, 3 & 11, 15 & 13, 20 & 2, 19 \\
    o, d & y, w & j, v & d, j & x, b & k, i & m, t & t, x & s, d & l, p & n, u & c, t \\
    \hline
\end{tabular} } \end{center}
\exampleend

Криптосистема Хилла уязвима к частотному криптоанализу\index{криптоанализ!частотный}, который основан на вычислении частот последовательностей символов. Рассмотрим пример <<взлома>> простого варианта криптосистемы Хилла.

\example В английском языке $m = 26$,
    \[ a \rightarrow 0, ~ b \rightarrow 1, ~ \dots, ~ z \rightarrow 25. \]
При шифровании использована криптосистема Хилла с матрицей второго порядка c нулевым вектором $\mathbf{f}$. Наиболее часто встречающиеся в шифртексте биграммы -- RH и NI, в то время как в исходном языке -- TH и HE (артикль THE). Найдём матрицу секретного ключа, составив уравнения
\[
    \begin{array}{l}
        R = 17 = -9 \mod 26, ~~ H = 7 \mod 26, ~~ N = 13 \mod 26, \\
        I = 8 \mod 26, ~~ T = 19 = -7 \mod 26, ~~ E=4 \mod 26; \\
    \end{array}
\] \[
    \left( \begin{array}{cc}
        \text{R} & \text{H} \\
        \text{N} & \text{I}
    \end{array} \right) =
    \left( \begin{array}{cc}
        \text{T} & \text{H} \\
        \text{H} & \text{E}
    \end{array} \right) \cdot
    \left( \begin{array}{cc}
        k_{1,1} & k_{1,2} \\
        k_{2,1} & k_{2,2}
    \end{array} \right) \mod 26;
\] \[
    \left( \begin{array}{cc}
        -9 & 7 \\
        13 & 8
    \end{array} \right) =
    \left( \begin{array}{cc}
        -7 & 7 \\
        7 & 4
    \end{array} \right) \cdot
    \left( \begin{array}{cc}
        k_{1,1} & k_{1,2} \\
        k_{2,1} & k_{2,2}
    \end{array} \right) \mod 26.
\]

Стоит обратить внимание на то, что числа 4, 8, 13 не имеют обратных элементов по модулю 26.
\[
    D = \det \left( \begin{array}{cc} -7 & 7 \\ 7 & 4 \end{array} \right) = -7 \cdot 4 - 7 \cdot 7 = 1 \mod 26.
\] \[
    \left( \begin{array}{cc} -7 & 7 \\ 7 & 4 \end{array} \right)^{-1} =
    D^{-1} \left( \begin{array}{cc} 4 & -7 \\ -7 & -7 \end{array} \right) =
    \left( \begin{array}{cc} 4 & -7 \\ -7 & -7 \end{array} \right) \mod 26.
\] \[
    \left( \begin{array}{cc} k_{1,1} & k_{1,2} \\ k_{2,1} & k_{2,2} \end{array} \right) =
    \left( \begin{array}{cc} 4 & -7 \\ -7 & -7 \end{array} \right) \cdot
    \left( \begin{array}{cc} -9 & 7 \\ 13 & 8 \end{array} \right) =
\] \[
    = \left( \begin{array}{cc} 3 & -2 \\ -2 & -1 \end{array} \right) \mod 26.
\]
Найденный секретный ключ
\[
    \left( \begin{array}{cc} \text{D} & \text{Y} \\ \text{Y} & \text{Z} \end{array} \right).
\]
\exampleend

\index{шифр!Хилла|)}
