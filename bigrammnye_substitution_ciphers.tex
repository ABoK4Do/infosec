\section{Биграммные шифры замены}\index{шифр!биграммный}
\selectlanguage{russian}

Если при шифровании преобразуются по две буквы открытого текста, то такой шифр называется \emph{биграммным}\index{шифр!биграммный} шифром замены. Первый биграммный шифр был изобретён аббатом Иоганном Тритемием и опубликован в 1508-м году. Другой биграммный шифр изобретён в 1854 году Чарльзом Витстоном. Лорд Лайон Плейфер (\langen{Lyon Playfair}) внедрил этот шифр в государственных службах Великобритании, и шифр был назван шифром Плейфера\index{шифр!Плейфера}.

Опишем шифр Плейфера\index{шифр!Плейфера}. Составляется таблица для английского алфавита (буквы \texttt{I}, \texttt{J} отождествляются), в которую заносятся буквы перемешанного алфавита, например, в виде таблицы, представленной ниже. Часто перемешивание алфавита реализуется с помощью начального слова, в котором отбрасываются повторяющиеся символы. В нашем примере начальное слово \texttt{playfair}. Таблица имеет вид:

\begin{center}
    \begin{tabular}{ccccc}
        p & l & a & y & f  \\
        i & r & b & c & d  \\
        e & g & h & k & m  \\
        n & o & q & s & t  \\
        u & v & w & x & z  \\
    \end{tabular}
\end{center}

Буквы открытого текста разбиваются на пары. Правила шифрования каждой пары состоят в следующем.

\begin{itemize}
    \item Если буквы пары не лежат в одной строке или в одном столбце таблицы, то они заменяются буквами, образующими с исходными буквами вершины прямоугольника. Первой букве пары соответствует буква таблицы, находящаяся в том же столбце. Пара букв открытого текста \texttt{we} заменяется двумя буквами таблицы \texttt{hu}. Пара букв открытого текста \texttt{ew} заменяется двумя буквами таблицы \texttt{uh}.
    \item Если буквы пары открытого текста расположены в одной строке таблицы, то каждая буква заменяется соседней справа буквой таблицы. Например, пара \texttt{gk} заменяется двумя буквами \texttt{hm}. Если одна из этих букв -- крайняя правая в таблице, то её <<правым соседом>> считается крайняя левая в этой строке. Так, пара \texttt{to} заменяется буквами \texttt{nq}.
    \item Если буквы пары лежат в одном столбце, то каждая буква заменяется соседней буквой снизу. Например, пара \texttt{lo} заменяется парой \texttt{rv}. Если одна из этих букв крайняя нижняя, то её <<нижним соседом>> считается крайняя верхняя буква в этом столбце таблицы. Например, пара \texttt{kx} заменяется буквами \texttt{sy}.
    \item Если буквы в паре одинаковые, то между ними вставляется определённая буква, называемая <<буквой-пустышкой>>. После этого разбиение на пары производится заново.
\end{itemize}

\example
Используем шифр Плейфера\index{шифр!Плейфера} и зашифруем сообщение "\texttt{Wheatstone was the inventor}". Исходное сообщение, разбитое на биграммы, показано в первой строке таблицы. Результат шифрования, также разбитый на биграммы, приведён во второй строке.
\begin{center} \begin{tabular}{|*{12}c|}
    \hline
    wh & ea & ts & to & ne & wa & st & he & in & ve & nt & or \\
    \hline
    aq & ph & nt & nq & un & ab & tn & kg & eu & gu & on & vg \\
    \hline
\end{tabular} \end{center}
\exampleend

Шифр Плейфера\index{шифр!Плейфера} не является криптографически стойким. Несложно найти ключ, если известны пара открытого текста и соответствующего ему шифртекста. Если известен только шифртекст, криптоаналитик может проанализировать соответствие между частотой появления биграмм в шифртексте и известной частотой появления биграмм в языке, на котором написано сообщение. Такой частотный анализ помогает дешифрованию.
