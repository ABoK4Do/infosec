\subsection{Обратимость схемы Фейстеля}
\selectlanguage{russian}

Покажем, что обратимость схемы Фейстеля не зависит от выбора функции $F$.

Напомним, что схема Фейстеля -- это итеративное шифрование, в котором выход подается на вход следующей итерации по правилу:
\[ \begin{array}{l}
    L_i = R_{i-1}, \\
    R_i = L_{i-1} \oplus F(R_{i-1}, K_i), \\
\end{array} \]
\[
    (L_0,R_0) \rightarrow (L_1,R_1) \rightarrow \ldots \rightarrow (L_n,R_n).
\]

При расшифровании используется та же схема, только левая и правая части меняются местами перед началом итераций, а ключи раунда подаются в обратном порядке
    \[ R_i = L_{i-1} \oplus F(R_{i-1}, K_{n+1-i}), \]
\[ \begin{array}{l}
    L_0^* = R_n = L_{n-1} \oplus F(R_{n-1}, K_n), \\
    R_0^* = L_n = R_{n-1}, \\
    \\
%\end{array} \]
%\[ \begin{array}{l}
    L_1^* = R_{n-1}, \\
    R_1^* = L_{n-1} \oplus F(R_{n-1}, K_n) \oplus F(R_{n-1}, K_n) = L_{n-1}, \\
    \dots
\end{array} \]
