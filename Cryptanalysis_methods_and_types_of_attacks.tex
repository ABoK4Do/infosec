\section{Методы криптоанализа и типы атак}
\selectlanguage{russian}

Нелегальный пользователь-криптоаналитик получает информацию путём дешифрования. Сложность этой процедуры определяется числом стандартных операций, которые надо выполнить для достижения цели. \emph{Двоичной сложностью}\index{сложность!двоичная} (или битовой сложностью) алгоритма называется количество двоичных операций, которые необходимо выполнить для его завершения.
% Наиболее сложным является дешифрование полиалфавитных шифров.

Рассмотрим основные сценарии работы криптоаналитика $E$. В первом сценарии криптоаналитик может осуществлять подслушивание и (или) перехват сообщений. Его вмешательство не нарушает целостности информации: $Y=\widetilde{Y}, где Y - целостность информации до вмешательства, \widetilde{Y} - целостность информации после вмешательства $. Эта роль криптоаналитика называется \emph{пассивной}. Так как он получает доступ к информации, то здесь нарушается конфиденциальность.

Во втором сценарии роль криптоаналитика \emph{активная}. Он может подслушивать, перехватывать сообщения и преобразовывать их по своему усмотрению: задерживать, искажать с помощью перестановок пакетов, устраивать обрыв связи, создавать новые сообщения и т.~п. В этом случае выполняется условие $Y \neq \widetilde{Y}$. Это значит, что одновременно нарушается целостность и конфиденциальность передаваемой информации.

Приведём примеры пассивных и активных атак:
\begin{itemize}
    \item Атака <<\emph{человек посередине}>>\index{атака!<<человек посередине>>} (\langen{man-in-the-middle}) подразумевает криптоаналитика, который разрывает канал связи, встраиваясь между $A$ и $B$, получает сообщения от $A$ и от $B$, а от себя отправляет новые, фальсифицированные сообщения. В результате $A$ и $B$ не замечают, что общаются с $E$, а не друг с другом.
    \item Атака \emph{воспроизведения}\index{атака!воспроизведения} (\langen{replay attack}) предполагает, что криптоаналитик может записывать и воспроизводить шифртексты, имитируя легального пользователя.
    \item Атака на \emph{различение} сообщений\index{атака!на различение} означает, что криптоаналитик, наблюдая одинаковые шифртексты, может извлечь информацию об идентичности исходных открытых текстов.
    \item Атака на \emph{расширение} сообщений\index{атака!на расширение} означает, что криптоаналитик может дополнить шифртекст осмысленной информацией без знания секретного ключа.
    \item \emph{Фальсификация} шифртекстов\index{атака!фальсификацией} криптоаналитиком без знания секретного ключа.
\end{itemize}

Часто для нахождения секретного ключа криптоатаки строят в предположениях о доступности дополнительной информации. Приведём примеры:
\begin{itemize}
    \item Атака на основе известного открытого текста\index{атака!с известным открытым текстом} (\langen{chosen plaintext attack, CPA}) предполагает, что криптоаналитик имеет возможность выбирать открытый текст и получать для него соответствующий шифртекст.
    \item Атака на основе известного шифртекста\index{атака!с известным шифртекстом} (\langen{chosen ciphertext attack, CCA}) предполагает возможность криптоаналитику выбирать шифртекст и получать для него соответствующий открытый текст.
\end{itemize}

Обязательным требованием к современным криптосистемам является устойчивость ко всем известным типам атак: пассивным, активным и с дополнительной информацией.


%Приведём примеры возможных вариантов работы активного криптоаналитика.
%\begin{itemize}
%\item Криптоаналитик имеет $m$ шифрованных сообщений $Y_{1},Y_{2},\ldots Y_{m}$ и пытается определить ключ или прочитать открытый текст $X_{1},X_{2},\ldots X_{m}.$
%\item Криптоаналитик имеет несколько пар открытого и шифрованного текстов
%
%$(Y_{1},X_{1}),(Y_{2}X_{2}),\ldots (Y_{m}X_{m})$ и пытается дешифровать остальной текст или определить алгоритм шифрования или определить ключ.
%\item
%\item
%\item
%\end{itemize}

Для защиты информации от активного криптоаналитика и обеспечения её целостности дополнительно к шифрованию сообщений применяют имитовставку\index{имитовставка}. Для неё используют обозначение $\MAC$ (\langen{message authentication code}). Как правило, $\MAC$ строится на основе хэш-функций, которые будут описаны далее.

Существуют ситуации, когда пользователи $A$ и $B$ не доверяют друг другу. Например, $A$ -- банк, $B$ -- получатель денег. $A$ утверждает, что деньги были переведены, $B$ - что перевода не было. Решение задачи аутентификации и неотрицаемости состоит в обеспечении \emph{электронной подписью}\index{электронная подпись} каждого из абонентов. Предварительно надо решить задачу о генерировании и распределении секретных ключей.

В общем случае, системы защиты информации должны обеспечивать:
\begin{itemize}
    \item конфиденциальность\index{конфиденциальность} (защиту от наблюдения),
    \item целостность\index{целостность} (защиту от изменения),
    \item аутентификацию (защиту от фальсификации пользователя и сообщений),
    \item доказательство авторства информации (доказательство авторства и защита от его отрицания)
\end{itemize}
как со стороны получателя, так и со стороны отправителя.

Важным критерием для выбора степени защиты является сравнение стоимости реализации взлома для получения информации и экономического эффекта от владения ей. Очевидно, что если стоимость взлома превышает ценность информации, взлом нецелесообразен.

%Сценарии защиты информации
%   Сценарий 1. A -- передающая сторона. B -- принимающая сторона. E -- пассивный
%криптоаналитик, который может подслушивать передачу, но не может вмешиваться
%в процесс передачи. Цель защиты: обеспечение конфиденциальности. Средства
%-- методы шифрования с секретным ключом (симметричные системы шифрования)
%и методы шифрования с открытым ключом (асимметричные системы шифрования).
%Сценарий 2. E -- активный криптоаналитик, который может изменять, удалять и вставлять
%сообщения или их части. Цель защиты -- обеспечение конфиденциальности (не
%всегда) и обеспечение целостности. Средства -- методы шифрования и добавление
%имитовставки\index{имитовставка} (Message Autentication Code -- $\MAC$).
%Сценарий 3. A и B не доверяют друг другу. Цель защиты -- аутентификация пользователя.
%Средства -- электронная подпись.
