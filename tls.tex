\section{Протокол SSL/TLS}\index{протокол!SSL/TLS}
\selectlanguage{russian}

Протокол SSL (Secure Sockets Layer) был разработан компанией Netscape. Начиная с версии 3 протокол развивается как открытый стандарт TLS (Transport Layer Security). Протокол SSL/TLS обеспечивает защищённое соединение по незащищённому каналу связи на прикладном уровне модели TCP/IP. Протокол встраивается между прикладным и транспортным уровнями стека протоколов TCP/IP. Для обозначения <<новых>> протоколов, полученных с помощью инкапсуляции прикладного уровня (HTTP\index{протокол!HTTP}, FTP\index{протокол!FTP}, SMTP\index{протокол!SMTP}, POP3\index{протокол!POP3}, IMAP\index{протокол!IMAP} и~т.\,д.) в SSL/TLS, к обозначению добавляют суффикс <<S>> (<<Secure>>): HTTPS\index{протокол!HTTPS}, FTPS\index{протокол!FTPS}, POP3S\index{протокол!POP3S}, IMAPS\index{протокол!IMAPS} и~т.\,д.

Протокол обеспечивает следующее:
\begin{itemize}
    \item Одностороннюю или взаимную аутентификацию клиента и сервера по открытым ключам сертификата X.509. В Интернете, как правило, делается \textbf{односторонняя} аутентификация веб-сервера браузеру клиента, то есть только веб-сервер предъявляет сертификат (открытый ключ и ЭП к нему от вышележащего УЦ).
    \item Создание сеансовых симметричных ключей для шифрования и кода аутентификации сообщения для передачи данных в обе стороны.
    \item Конфиденциальность\index{конфиденциальность} -- блочное или потоковое шифрование передаваемых данных в обе стороны.
    \item Целостность\index{целостность} -- аутентификацию отправляемых сообщений в обе стороны имитовставкой\index{имитовставка} $\HMAC(K,M)$, описанной ранее.
\end{itemize}

Рассмотрим протокол TLS последней версии 1.2.


\subsection{Протокол <<рукопожатия>>}

Протокол <<рукопожатия>> (Handshake Protocol) производит аутентификацию и создание сеансовых ключей между клиентом $C$ и сервером $S$.

\begin{enumerate}
    \item $C \rightarrow S$:
        \begin{enumerate}
            \item ClientHello: ~ 1) URI сервера, ~ 2) одноразовая метка $N_C$\index{одноразовая метка}, ~ 3) поддерживаемые алгоритмы шифрования, кода аутентификации сообщений, хэширования, ЭП и сжатия.
        \end{enumerate}

    \item $C \leftarrow S$:
        \begin{enumerate}
            \item ServerHello: одноразовая метка $N_S$, поддерживаемые алгоритмы сервера.

            После обмена набором желательных алгоритмов сервер и клиент по единому правилу выбирают общий набор алгоритмов.
            \item Server Certificate: сертификат X.509v3 сервера с запрошенным URI (URI нужен в случае нескольких виртуальных веб-серверов с разными URI на одном узле c одним IP-адресом).
            \item Server Key Exchange Message: информация для создания предварительного общего секрета $premaster$ длиной 48 байтов в виде: ~ 1) обмена по протоколу Диффи~---~Хеллмана\index{протокол!Диффи~---~Хеллмана} с клиентом (сервер отсылает $(g, g^a)$), ~ 2) по другому алгоритму с открытым ключом, ~ 3) разрешения клиенту выбрать ключ.
            \item Электронная подпись к Server Key Exchange Message на ключе сертификата сервера для аутентификации сервера клиенту.
            \item Certificate Request: опциональный запрос сервером сертификата клиента.
            \item Server Hello Done: идентификатор конца транзакции.
        \end{enumerate}

    \item $C \rightarrow S$:
        \begin{enumerate}
            \item Client Certificate: сертификат X.509v3 клиента, если он был запрошен сервером.
            \item Client Key Exchange Message: информация для создания предварительного общего секрета $premaster$ длиной 48 байтов в виде: ~ 1) либо обмена по протоколу Диффи~---~Хеллмана\index{протокол!Диффи~---~Хеллмана} с сервером (клиент отсылает $g^b$, в результате обе стороны вычисляют ключ $premaster = g^{ab}$), ~ 2) либо по другому алгоритму, ~ 3) либо ключа, выбранного клиентом и зашифрованного на открытом ключе из сертификата сервера.
            \item Электронная подпись к Client Key Exchange Message на ключе сертификата клиента для аутентификации клиента серверу (если клиент использует сертификат).
            \item Certificate Verify: результат проверки сертификата сервера.
            \item Change Cipher Spec: уведомление о смене сеансовых ключей.
            \item Finished: идентификатор конца транзакции.
        \end{enumerate}

    \item $C \leftarrow S$:
        \begin{enumerate}
            \item Change Cipher Spec: уведомление о смене сеансовых ключей.
            \item Finished: идентификатор конца транзакции.
        \end{enumerate}
\end{enumerate}

%      http://tools.ietf.org/html/rfc5246#page-37

%      struct {
%          ProtocolVersion client_version;
%          Random random;
%          SessionID session_id;
%          CipherSuite cipher_suites<2..2^16-2>;
%          CompressionMethod compression_methods<1..2^8-1>;
%          select (extensions_present) {
%              case false:
%                  struct {};
%              case true:
%                  Extension extensions<0..2^16-1>;
%          };
%      } ClientHello;

%      struct {
%          ProtocolVersion server_version;
%          Random random;
%          SessionID session_id;
%          CipherSuite cipher_suite;
%          CompressionMethod compression_method;
%          select (extensions_present) {
%              case false:
%                  struct {};
%              case true:
%                  Extension extensions<0..2^16-1>;
%          };
%      } ServerHello;

%      struct {
%          ASN.1Cert certificate_list<0..2^24-1>;
%      } Certificate;

%      struct {
%          select (KeyExchangeAlgorithm) {
%              case dh_anon:
%                  ServerDHParams params;
%              case dhe_dss:
%              case dhe_rsa:
%                  ServerDHParams params;
%                  digitally-signed struct {
%                      opaque client_random[32];
%                      opaque server_random[32];
%                      ServerDHParams params;
%                  } signed_params;
%              case rsa:
%              case dh_dss:
%              case dh_rsa:
%                  struct {} ;
%                 /* message is omitted for rsa, dh_dss, and dh_rsa */
%              /* may be extended, e.g., for ECDH -- see [TLSECC] */
%          };
%      } ServerKeyExchange;

%      struct {
%          ClientCertificateType certificate_types<1..2^8-1>;
%          SignatureAndHashAlgorithm
%            supported_signature_algorithms<2^16-1>;
%          DistinguishedName certificate_authorities<0..2^16-1>;
%      } CertificateRequest;

%      struct {
%          select (KeyExchangeAlgorithm) {
%              case rsa:
%                  EncryptedPreMasterSecret;
%              case dhe_dss:
%              case dhe_rsa:
%              case dh_dss:
%              case dh_rsa:
%              case dh_anon:
%                  ClientDiffieHellmanPublic;
%          } exchange_keys;
%      } ClientKeyExchange;

%      struct {
%           digitally-signed struct {
%               opaque handshake_messages[handshake_messages_length];
%           }
%      } CertificateVerify;

%      struct {
%          opaque verify_data[verify_data_length];
%      } Finished;

Одноразовая метка $N_C$ состоит из 32 байтов. Первые 4 байта содержат текущее время (gmt\_unix\_time), оставшиеся байты -- псевдослучайную последовательность, которую формирует криптографически стойкий генератор псевдослучайных чисел.

Предварительный общий секрет $premaster$ длиной 48 байтов вместе с одноразовыми метками используется как инициализирующее значение генератора $PRF$ для получения общего секрета $master$ тоже длиной 48 байтов:
    \[ master = PRF(premaster, ~\text{текст ''master secret''}, ~ N_C + N_S) .\]

И, наконец, уже из секрета $master$ таким же способом генерируется 6 окончательных сеансовых ключей, следующих друг за другом в битовой строке:
    \[ \{ (K_{E,1} ~\|~ K_{E,2}) ~\|~ (K_{\MAC,1} ~\|~ K_{\MAC,2}) ~\|~ (IV_1 ~\|~ IV_2) \} = \]
        \[ = PRF(master, ~\text{текст ''key expansion''}, ~ N_C + N_S), \]
где $K_{E,1}, ~ K_{E,2}$ -- два ключа симметричного шифрования, ~ $K_{\MAC,1}, ~ K_{\MAC,2}$ -- два ключа имитовставки\index{имитовставка}, ~ $IV_1, ~IV_2$ -- два инициализирующих вектора режима сцепления блоков\index{вектор инициализации}. Ключи с индексом 1 используются для коммуникации от клиента к серверу, с индексом 2 -- от сервера к клиенту.


\subsection{Протокол записи}

Протокол записи (Record Protocol) определяет формат TLS-пакетов для вложения в TCP-пакеты.

\begin{enumerate}
    \item Исходными сообщениями $M$ для шифрования являются пакеты протокола следующего уровня в модели OSI: HTTP\index{протокол!HTTP}, FTP\index{протокол!FTP}, IMAP\index{протокол!IMAP} и~т.\,д.
    \item Сообщение $M$ разбивается на блоки $m_i$ не более 16 кБ.
    \item Блоки $m_i$ сжимаются алгоритмом компрессии в блоки $z_i$.
    \item Вычисляется имитовставка\index{имитовставка} для каждого блока $z_i$ и добавляется в конец блоков: $a_i = z_i ~\|~ \HMAC(K_{\MAC}, z_i)$.
    \item Блоки $a_i$ шифруются симметричным алгоритмом с ключом $K_E$ в некотором режиме сцепления блоков с инициализирующим вектором $IV$ в полное сжатое аутентифицированное зашифрованное сообщение $C$.
    \item К шифротексту $C$ добавляется заголовок протокола записи TLS, и в результате получается TLS-пакет для вложения в TCP-пакет.
\end{enumerate}
