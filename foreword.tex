\chapter*{Предисловие}
\selectlanguage{russian}
\addcontentsline{toc}{chapter}{Предисловие}
Изучение курса <<Защита информации>> необходимо начать с определения понятия \emph{информация}. В теоретической информатике \textbf{информация} -- это любые сведения, или цифровые данные, или сообщения, или документы, или файлы, которые могут быть переданы \emph{получателю информации} от \emph{источника информации}. Можно считать, что информация передается по какому-либо каналу связи с помощью некоторого носителя, которым может быть, например, распечатка текста, диск или другое устройство хранения информации, система передачи сигналов по оптическим, проводным или радио линиям связи и т. д.

\textbf{Защита информации} -- это сохранение \emph{целостности}, \emph{конфиденциальности} и \emph{доступности} информации, передаваемой или хранимой в какой-либо форме. Информацию необходимо защищать от разрушения ее целостности и конфиденциальности в результате  вмешательства \emph{нелегального пользователя}.

\textbf{Целостность информации}\index{целостность} -- это сохранность информации (в любой ее форме представления) в неизменном (оригинальном) виде. \textbf{Конфиденциальность}\index{конфиденциальность} означает, что информация получена именно тем, кому она предназначалась, то есть \textbf{легальным пользователем}, и никто другой, то есть \emph{нелегальный пользователь}, эту информацию не получил.

\textbf{Доступность}\index{доступность} -- свойство информации быть доступной легальным пользователям в любой момент времени.

Чтобы реализовать защиту информации, используются различные математические методы, технические средства и организационные меры. В частности, источник информации (на передающей стороне) применяет \textbf{шифрование}, а  легальный пользователь (на приемной стороне) осуществляет \textbf{расшифрование}\index{расшифрование}. Процесс получения информации нелегальным пользователем называется \textbf{дешифрованием}\index{дешифрование}\footnote{В англоязычной литературе словом <<decryption>> обозначается и расшифрование, и дешифрование.}, а сам нелегальный пользователь -- \textbf{криптоаналитиком}\index{криптоаналитик}.

В настоящем пособии рассмотрены только основные математические методы защиты информации, и среди них основной акцент сделан на криптографическую защиту, которая включает симметричные и несимметричные методы шифрования, формирование  секретных ключей, протоколы ограничения доступа и аутентификации сообщений и пользователей. Кроме того, в пособии рассматриваются типовые уязвимости операционных и информационно-вычислительных систем.
