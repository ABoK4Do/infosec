\chapter*{Предисловие}
\addcontentsline{toc}{chapter}{Предисловие}
\selectlanguage{russian}

Изучение курса <<Защита информации>> необходимо начать с определения понятия \emph{<<информация>>}. В теоретической информатике \textbf{информация} -- это любые сведения, или цифровые данные, или сообщения, или документы, или файлы, которые могут быть переданы \emph{получателю информации} от \emph{источника информации}. Можно считать, что информация передаётся по какому-либо каналу связи с помощью некоторого носителя, которым может быть, например, распечатка текста, диск или другое устройство хранения информации, система передачи сигналов по оптическим, проводным или радио линиям связи и т.~д.

\textbf{Защита информации} -- это сохранение \emph{целостности}, \emph{конфиденциальности} и \emph{доступности} информации, передаваемой или хранимой в какой-либо форме. Информацию необходимо защищать от разрушения её целостности и конфиденциальности в результате вмешательства \emph{нелегального пользователя}.

\textbf{Целостность информации}\index{целостность} -- это сохранность информации (в любой её форме представления) в неизменном (оригинальном) виде. \textbf{Конфиденциальность}\index{конфиденциальность} означает, что информация получена именно тем, кому она предназначалась, то есть \textbf{легальным пользователем}, и никто другой, то есть \emph{нелегальный пользователь}, эту информацию не получил.

\textbf{Доступность}\index{доступность} -- свойство информации быть доступной легальным пользователям в любой момент времени.

Чтобы реализовать защиту информации, используются различные математические методы, технические средства и организационные меры. В частности, источник информации (на передающей стороне) применяет \emph{шифрование}, а легальный пользователь (на приёмной стороне) осуществляет \emph{расшифрование}\index{расшифрование}. Процесс получения информации нелегальным пользователем называется \emph{дешифрованием}\index{дешифрование}\footnote{В англоязычной литературе словом <<decryption>> обозначается и расшифрование, и дешифрование.}, а сам нелегальный пользователь -- \emph{криптоаналитиком}\index{криптоаналитик}.

В настоящем пособии рассмотрены только основные математические методы защиты информации, и среди них основной акцент сделан на криптографическую защиту, которая включает симметричные и несимметричные методы шифрования, формирование секретных ключей, протоколы ограничения доступа и аутентификации сообщений и пользователей. Кроме того, в пособии рассматриваются типовые уязвимости операционных и информационно-вычислительных систем.

\section*{Благодарности}
\addcontentsline{toc}{section}{Благодарности}
Авторы пособия благодарят студентов, аспирантов и сотрудников института, которые помогли с подготовкой, редактированием и поиском ошибок в тексте:

\begin{itemize}
	\item Дмитрий Банков (201-011 гр.)
	\item Даниил Бершацкий (201-012 гр.)
	\item Дмитрий Бородий (201-112 гр.)
	\item Эмиль Вахитов (201-114 гр.)
	\item Дмитрий Вербицкий (201-119 гр.)
	\item Тагир Гадельшин (201-119 гр.)
	\item Марат Гаджибутаев (201-018 гр.)
	\item Ильназ Гараев (201-113 гр.)
	\item Евгений Глушков (201-012 гр.)
	\item Сергей Жестков (201-013 гр.)
	\item Дмитрий Зборовский (201-119 гр.)
	\item Марат Ибрагимов (201-114 гр.)
	\item Александр Иванов (201-011 гр.)
	\item Александр Иванов (201-019 гр.)
	\item Атнер Иванов (201-114 гр.)
	\item Владимир Ивашкин (201-112 гр.)
	\item Ирина Камалова (201-115 гр.)
	\item Иван Киселев (201-115 гр.)
	\item Константин Ковальков (201-015 гр.)
	\item Александр Кравцов (201-116 гр.)
	\item Виталий Крепак (201-013 гр.)
	\item Александр Кротов (201-011 гр.)
	\item Станислав Круглик (201-111 гр.)
	\item Зулкаид Курбанов (201-113 гр.)
	\item Алексей Мамаков (201-113 гр.)
	\item Дао Куанг Минь (201-116 гр.)
	\item Надежда Мозолина (201-119 гр.)
	\item Милосердов Олег (201-016 гр.)
	\item Хыу Чунг Нгуен (201-015 гр.)
	\item Артём Никитин (201-012 гр.)
	\item Андрей Пунь (201-013 гр.)
	\item Вадим Сафронов (201-112 гр.)
	\item Иван Саюшев (201-112 гр.)
	\item Игорь Сорокин (201-112 гр.)
	\item Буй Зуи Тан (201-112 гр.)
	\item Татьяна Тюпина (201-116 гр.)
	\item Сергей Угрюмов (201-119 гр.)
	\item Марсель Файзуллин (201-114 гр.)
	\item Нияз Фазлыев (201-114 гр.)
	\item Евгений Юлюгин (201-916 гр.)
\end{itemize}
