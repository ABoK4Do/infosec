\chapter*{Предисловие}
\addcontentsline{toc}{chapter}{Предисловие}
\selectlanguage{russian}

Изучение курса <<Защита информации>> необходимо начать с определения понятия \emph{<<информация>>}. В теоретической информатике \textbf{информация} -- это любые сведения, или цифровые данные, или сообщения, или документы, или файлы, которые могут быть переданы \emph{получателю информации} от \emph{источника информации}. Можно считать, что информация передаётся по какому-либо каналу связи с помощью некоторого носителя, которым может быть, например, распечатка текста, диск или другое устройство хранения информации, система передачи сигналов по оптическим, проводным линиям или радиолиниям связи и т.~д.

\textbf{Защита информации} -- это сохранение \emph{целостности}, \emph{конфиденциальности} и \emph{доступности} информации, передаваемой или хранимой в какой-либо форме. Информацию необходимо защищать от разрушения её целостности и конфиденциальности в результате вмешательства \emph{нелегального пользователя}.

\textbf{Целостность информации}\index{целостность} -- это сохранность информации (в любой её форме представления) в неизменном (оригинальном) виде. \textbf{Конфиденциальность}\index{конфиденциальность} означает, что информация получена именно тем, кому она предназначалась, то есть \textbf{легальным пользователем}, и никто другой, то есть \emph{нелегальный пользователь}, эту информацию не получил.

\textbf{Доступность}\index{доступность} -- свойство информации быть доступной легальным пользователям в любой момент времени.

Чтобы реализовать защиту информации, используются различные математические методы, технические средства и организационные меры. В частности, источник информации (на передающей стороне) применяет \emph{шифрование}, а легальный пользователь (на приёмной стороне) осуществляет \emph{расшифрование}\index{расшифрование}. Процесс получения информации нелегальным пользователем называется \emph{дешифрованием}\index{дешифрование}\footnote{В англоязычной литературе словом <<decryption>> обозначается и расшифрование, и дешифрование.}, а сам нелегальный пользователь -- \emph{криптоаналитиком}\index{криптоаналитик}.

В настоящем пособии рассмотрены только основные математические методы защиты информации, и среди них основной акцент сделан на криптографическую защиту, которая включает симметричные и несимметричные методы шифрования, формирование секретных ключей, протоколы ограничения доступа и аутентификации сообщений и пользователей. Кроме того, в пособии рассматриваются типовые уязвимости операционных и информационно-вычислительных систем.

\section*{Благодарности}
\addcontentsline{toc}{section}{Благодарности}
Авторы пособия благодарят студентов, аспирантов и сотрудников института, которые помогли с подготовкой, редактированием и поиском ошибок в тексте:

\begin{multicols}{2}
\begin{small}
\begin{enumerate*}\itemsep1pt \parskip0pt \parsep0pt
	\item Татьяна Бакланова\begin{tiny} (201-211 гр.)\end{tiny}
	\item Дмитрий Банков\begin{tiny} (201-011 гр.)\end{tiny}
	\item Даниил Бершацкий\begin{tiny} (201-012 гр.)\end{tiny}
	\item Дмитрий Бородий\begin{tiny} (201-112 гр.)\end{tiny}
	\item Илья Васильев\begin{tiny} (201-217 гр.)\end{tiny}
	\item Эмиль Вахитов\begin{tiny} (201-114 гр.)\end{tiny}
	\item Дмитрий Вербицкий\begin{tiny} (201-119 гр.)\end{tiny}
	\item Тагир Гадельшин\begin{tiny} (201-119 гр.)\end{tiny}
	\item Марат Гаджибутаев\begin{tiny} (201-018 гр.)\end{tiny}
	\item Ильназ Гараев\begin{tiny} (201-113 гр.)\end{tiny}
	\item Евгений Глушков\begin{tiny} (201-012 гр.)\end{tiny}
	\item Андрей Горбунов\begin{tiny} (201-116 гр.)\end{tiny}
	\item Алексей Гусаров\begin{tiny} (201-216 гр.)\end{tiny}
	\item Сергей Жестков\begin{tiny} (201-013 гр.)\end{tiny}
	\item Виталий Занкин\begin{tiny} (201-111 гр.)\end{tiny}
	\item Дмитрий Зборовский\begin{tiny} (201-119 гр.)\end{tiny}
	\item Марат Ибрагимов\begin{tiny} (201-114 гр.)\end{tiny}
	\item Александр Иванов\begin{tiny} (201-011 гр.)\end{tiny}
	\item Александр Иванов\begin{tiny} (201-019 гр.)\end{tiny}
	\item Атнер Иванов\begin{tiny} (201-114 гр.)\end{tiny}
	\item Владимир Ивашкин\begin{tiny} (201-112 гр.)\end{tiny}
	\item Ирина Камалова\begin{tiny} (201-115 гр.)\end{tiny}
	\item Иван Киселёв\begin{tiny} (201-115 гр.)\end{tiny}
	\item Константин Ковальков\begin{tiny} (201-015 гр.)\end{tiny}
	\item Андрей Кочетыгов\begin{tiny} (201-111 гр.)\end{tiny}
	\item Александр Кравцов\begin{tiny} (201-116 гр.)\end{tiny}
	\item Виталий Крепак\begin{tiny} (201-013 гр.)\end{tiny}
	\item Александр Кротов\begin{tiny} (201-011 гр.)\end{tiny}
	\item Станислав Круглик\begin{tiny} (201-111 гр.)\end{tiny}
	\item Егор Кузнецов\begin{tiny} (201-211 гр.)\end{tiny}
	\item Зулкаид Курбанов\begin{tiny} (201-113 гр.)\end{tiny}
	\item Егор Макарычев\begin{tiny} (201-115 гр.)\end{tiny}
	\item Ольга Малюгина\begin{tiny} (201-111 гр.)\end{tiny}
	\item Алексей Мамаков\begin{tiny} (201-113 гр.)\end{tiny}
	\item Роман Маракулин\begin{tiny} (201-211 гр.)\end{tiny}
	\item Даниил Меркулов\begin{tiny} (201-111 гр.)\end{tiny}
	\item Олег Милосердов\begin{tiny} (201-016 гр.)\end{tiny}
	\item Дао Куанг Минь\begin{tiny} (201-116 гр.)\end{tiny}
	\item Антон Митрохин\begin{tiny} (201-216 гр.)\end{tiny}
	\item Надежда Мозолина\begin{tiny} (201-119 гр.)\end{tiny}
	\item Хыу Чунг Нгуен\begin{tiny} (201-015 гр.)\end{tiny}
	\item Артём Никитин\begin{tiny} (201-012 гр.)\end{tiny}
	\item Евгения Никольская\begin{tiny} (201-115 гр.)\end{tiny}
	\item Андрей Пунь\begin{tiny} (201-013 гр.)\end{tiny}
	\item Вадим Сафронов\begin{tiny} (201-112 гр.)\end{tiny}
	\item Иван Саюшев\begin{tiny} (201-112 гр.)\end{tiny}
	\item Илья Соломатин\begin{tiny} (201-211 гр.)\end{tiny}
	\item Игорь Сорокин\begin{tiny} (201-112 гр.)\end{tiny}
	\item Игорь Степанов\begin{tiny} (201-213 гр.)\end{tiny}
	\item Виктор Сухарев\begin{tiny} (201-114 гр.)\end{tiny}
	\item Буй Зуи Тан\begin{tiny} (201-112 гр.)\end{tiny}
	\item Татьяна Тюпина\begin{tiny} (201-116 гр.)\end{tiny}
	\item Сергей Угрюмов\begin{tiny} (201-119 гр.)\end{tiny}
	\item Марсель Файзуллин\begin{tiny} (201-114 гр.)\end{tiny}
	\item Нияз Фазлыев\begin{tiny} (201-114 гр.)\end{tiny}
	\item Наталья Федотова\begin{tiny} (201-212 гр.)\end{tiny}
	\item Данил Филиппов\begin{tiny} (201-115 гр.)\end{tiny}
	\item Александра Цветкова\begin{tiny} (201-216 гр.)\end{tiny}
	\item Евгений Юлюгин\begin{tiny} (201-916 гр.)\end{tiny}
	\item Руслан Юсупов\begin{tiny} (201-211 гр.)\end{tiny}
\end{enumerate*}
\end{small}
\end{multicols}