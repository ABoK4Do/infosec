\chapter{Совершенная криптостойкость}\index{криптостойкость!совершенная|(}
\selectlanguage{russian}

Рассмотрим модель криптосистемы, в которой Алиса выступает источником сообщений $m \in \group{M}$. Алиса использует некоторую функцию шифрования, результатом вычисления которой является шифртекст $c \in \group{C}$:

	\[c = E_{K_1}\left(m\right).\]

Шифртекст $c$ передаётся по открытому каналу легальному пользователю Бобу, причём по пути он может быть перехвачен нелегальным пользователем (криптоаналитиком) Евой.

Боб, обладая ключом расшифрования $K_2$, расшифровывает сообщение с использованием функции расшифрования:
	\[m' = D_{K_2}\left(c \right).\]

Рассмотрим теперь исходное сообщение, передаваемый шифртекст и ключи шифрования (и расшифрования, если они отличаются) в качестве случайных величин, описывая их свойства с точки зрения теории информации. Далее полагаем, что в криптосистеме ключи шифрования и расшифрования совпадают.

Будем называть криптосистему \emph{корректной}, если она обладает следующими свойствами:
\begin{itemize}
	\item легальный пользователь имеет возможность однозначно восстановить исходное сообщение, то есть:
					\[H \left( M | C K \right) = 0, \]
					\[m' = m\]
	\item выбор ключа шифрования не зависит от исходного сообщения:
					\[ I \left( K ; M \right) = 0, \]
					\[ H \left( K | M \right) = H \left( K \right). \]
\end{itemize}

Второе свойство является в некотором виде условием на возможность отделить ключ шифрования от данных и алгоритма шифрования.

\section[Определения]{Определения совершенной криптостойкости}

Понятие совершенной секретности (или стойкости) введено американским учёным Клодом Шенноном. В конце Второй мировой войны он закончил работу, посвящённую теории связи в секретных системах~\cite{Shannon:1949:CTS}. Эта работа вошла составной частью в собрание его трудов, вышедшее в русском переводе в 1963 году~\cite{Shannon:1963}. Понятие о стойкости шифров по Шеннону связано с решением задачи криптоанализа по одной криптограмме.

Криптосистемы совершенной стойкости могут применяться как в современных вычислительных сетях, так и для шифрования любой бумажной корреспонденции. Основной проблемой применения данных шифров для шифрования больших объёмов информации является необходимость распространения ключей объёмом не меньшим, чем передаваемые сообщения.

\begin{definition}\label{perfect_by_probabilities}
Будем называть криптосистему \emph{совершенно криптостойкой}, если апостериорное распределение вероятностей исходного случайного сообщения $m_i \in \group{M}$ при регистрации случайного шифртекста $c_k \in \group{C}$ совпадает с априорным распределением~\cite{Gultyaeva:2010}:

	\[\forall m_j \in \group{M}, c_k \in \group{C} \hookrightarrow P \left( m = m_j | c = c_k \right) = P \left( m = m_j \right).\]
\end{definition}

Данное условие можно переформулировать в терминах статистических свойств сообщения, ключа и шифртекста как случайных величин.

\begin{definition}\label{perfect_by_enthropy}
Будем называть криптосистему \emph{совершенно криптостойкой}, если условная энтропия сообщения\index{энтропия!условная}\index{энтропия!открытого текста} при известном шифртексте равна безусловной:
\begin{gather*}
	H \left( M | C \right) = H \left( M \right),\\
	I \left( M; C \right) = 0.
\end{gather*}
\end{definition}

Можно показать, что определения~\ref{perfect_by_probabilities} и~\ref{perfect_by_enthropy} тождественны.

\section[Условие]{Условие совершенной криптостойкости}

Найдём оценку количества информации, которое содержит шифртекст $C$ относительно сообщения $M$:
\[ I(M; C) = H(M) - H(M | C). \]
Очевидны следующие соотношения условных и безусловных энтропий~\cite{GabPil:2007}:
\begin{gather*}
H(K|C)=H(K|C)+H(M|KC)=H(MK|C),\\
H(MK|C)=H(M|C)+H(K|MC)\geq H(M|C),\\
H(K)\geq H(K|C)\geq H(M|C).
\end{gather*}
Отсюда получаем:
 \[ I(M; C) = H(M) - H(M | C)\geq H(M)-H(K). \]
Из последнего неравенства следует, что взаимная информация между сообщением и шифртекстом равна нулю, если энтропия ключа не меньше энтропии сообщений. С другой стороны, взаимная информация между сообщением и шифртекстом равна нулю, если они статистически независимы. Таким образом, условием совершенной криптостойкости является неравенство:
\[ H(M) \leq H(K).\]
%Если утверждение верно, то количество информации в шифртексте относительно открытого текста $I(M; C)$ равно нулю:
%  \[ I(M; C) = H(M) - H(M | C) = 0, \]
%так как для статистически независимых величин условная энтропия равна безусловной энтропии, то есть $H(M) = H(M | C)$.

%Функцию шифрования обозначим $E: \{ M, K \} \rightarrow C$. Процедура шифрования состоит из следующих шагов.
%\begin{itemize}
%    \item Легальный пользователь $A$ выбирает ключ $k \in K$ и секретно сообщает его легальному пользователю $B$ (дополнительная задача -- распределение ключей).
%    \item По открытому сообщению $m \in M$ и выбранному ключу $k$ вычисляют шифрованное сообщение $c = E_k(m) \in C$.
%\end{itemize}

%Основное требование при шифровании состоит в том, чтобы при выбранном ключе $k$ вычисление $c$ было лёгкой задачей для любого сообщения $m$.

%Функцию расшифрования обозначим $D: \{ C, K \} \rightarrow M$. Процедура расшифрования состоит из следующих шагов.
%\begin{itemize}
%    \item Легальный пользователь $B$ получает от $A$ секретный ключ $k \in K$.
 %   \item $B$ по принятому шифрованному сообщению $c \in C$ и известному ключу $k$ вычисляет открытое сообщение $m = D_k(c) \in M$.
%\end{itemize}

%Основное требование: при выбранном ключе $k$ вычисление $m$ должно быть лёгкой задачей для любого $c$. С другой стороны, при неизвестном ключе $k$ вычисление открытого сообщения $m$ по известному шифрованному сообщению $c$ должно быть трудной задачей для любого $c$.

%Криптостойкость шифра оценивается числом операций, необходимым для определения: открытого текста $m$ по шифртексту $c$, либо ключа шифрования $k$ по открытому тексту $m$ и шифртексту $c$.

%$M, C, K$ интерпретируются как случайные величины.
%Пусть заданы распределения вероятностей $P_m(M), P_c(C), P_k(K)$. По определению шифрование $C = E_K(M)$ -- детерминированная функция своих аргументов.
%Если при выбранном шифре оказалось, что открытый текст $M$ и шифртекст $C$ -- статистически независимые случайные величины, то считается, что такая система обладает совершенной криптостойкостью.


%\subsection{Длина ключа}

%Пусть сообщения $m\in M$ и ключи $r\in K$ являются независимыми случайными величинами. Это значит, что их совместная вероятность $P_{mk}(M, K)$ равна произведению отдельных вероятностей:
%\[P_{mk}(M, K) = P_m(M) \cdot P_k(K).\]
%Пусть $C = E_K(M)$ -- множество шифрованных текстов, $M = D_K(C)$ -- множество расшифрованных текстов. Можно найти вероятности $P_c(C), P_{mck}(M,C,K)$.

%Используя известные соотношения о безусловной и условной энтропии~\cite{GabPil:2007}, оценим энтропию открытых текстов $M$ с учётом статистической независимости $M$ и $C$:
 %   \[ H(M) = H(M | C) \leq H(MK | C) = H(K | C) + H(M | CK) = \]     \[ = H(K | C) \leq H(K). \]

%Так как энтропия открытого текста при заданном шифртексте и известном ключе равна нулю, то $H(M|CK)=0$. В результате получаем     \[ H(M) \leq H(K). \]

Обозначим длины сообщения и ключа как $L(M)$ и $L(K)$ соответственно. Известно, что $H(M) \leq L(M)$~\cite{GabPil:2007}. Равенство $H(M) = L(M)$ достигается, когда сообщения состоят из статистически независимых и равновероятных символов. Такое же свойство выполняется и для случайных ключей $H(K) \leq L(K)$. Таким образом, достаточным условием совершенной криптостойкости системы можно считать неравенство
 \[ L(M) \leq L(K)\]
при случайном выборе ключа.

%С другой стороны, энтропия открытого текста $H(M)$ характеризует длину последовательности для описания случайной величины $M$ (открытого сообщения), а $H(K)$ характеризует длину последовательности для описания ключа. Следовательно, совершенная криптостойкость возможна только тогда, когда длина ключа не меньше, чем длина шифруемого сообщения, то есть
%\[
%H(M) \leq H(K).
%\]
%Как правило, длина сообщения заранее неизвестна и ограничена большим числом. Выбрать ключ длины не меньшей, чем возможное сообщение не представляется возможным или рациональным, и один и тот же ключ (или его преобразования) используется многократно для шифрования блоков сообщения фиксированной длины. То есть, $H(K) \ll H(M)$.

На самом деле, сообщение может иметь произвольную (заранее не ограниченную) длину. Поэтому генерация и главным образом доставка легальным пользователям случайного и достаточно длинного ключа становятся критическими проблемами. Практическим решением этих проблем является многократное использование одного и того же ключа при условии, что его длина гарантирует вычислительную невозможность любой известной атаки на подбор ключа.

\index{криптостойкость!совершенная|)}

\section{Криптосистема Вернама}\index{криптосистема!Вернама|(}

Приведём пример системы с совершенной криптостойкостью.

Пусть сообщение представлено двоичной последовательностью длины $N$:
    \[ m = (m_1, m_2, \dots, m_N). \]
Распределение вероятностей сообщений $P_m(m)$ может быть любым. Ключ также представлен двоичной последовательностью $ k = (k_1, k_2, \dots, k_N)$ той же длины, но с равномерным распределением
    \[ P_k(k) = \frac{1}{2^N} \]
для всех ключей.

Шифрование в криптосистеме Вернама осуществляется путём покомпонентного суммирования по модулю алфавита последовательностей открытого текста и ключа:
    \[ C = M \oplus K = (m_1 \oplus k_1, ~ m_2 \oplus k_2, \dots, m_N \oplus k_N). \]

Легальный пользователь знает ключ и осуществляет расшифрование:
    \[ M =C \oplus K = (c_1 \oplus k_1, ~ c_2 \oplus k_2, \dots, c_N \oplus k_N). \]

Найдём вероятностное распределение $N$-блоков шифртекстов, используя формулу
\begin{multline*}
P(c = a) = P(m \oplus k = a) = \sum_{m} P(m) P(m \oplus k = a | m) = \\
= \sum_{m} P(m) P(k \oplus m) = \sum_{m} P(m) \frac{1}{2^N} = \frac{1}{2^N}. 
\end{multline*}


Получили подтверждение известного факта: сумма двух случайных величин, одна из которых имеет равномерное распределение, является случайной величиной с равномерным распределением. В нашем случае распределение ключей равномерное, поэтому распределение шифртекстов тоже равномерное.

Запишем совместное распределение открытых текстов и шифртекстов:
    \[ P(m = a, c = b) ~=~ P(m = a) ~ P(c = b | m = a). \]

Найдём условное распределение:
\begin{multline*}
    P(c = b | m = a) = P(m \oplus k = b | m = a) = \\
    = P(k = b \oplus a | m = a) = P(k = b \oplus a) = \frac{1}{2^N},
\end{multline*}
так как ключ и открытый текст являются независимыми случайными величинами. Итого:
    \[ P(c=b | m=a) = \frac{1}{2^N}. \]

Подстановка правой части этой формулы в формулу для совместного распределения даёт
    \[ P(m=a,c=b)=P(m=a)\frac{1}{2^N}, \]
что доказывает независимость шифртекстов и открытых текстов в этой системе. По доказанному выше, количество информации в шифртексте относительно открытого текста равно нулю. Это значит, что рассмотренная криптосистема Вернама обладает совершенной секретностью (криптостойкостью) при условии, что для каждого $N$-блока (сообщения) генерируется случайный (одноразовый) $N$-ключ.

\index{криптосистема!Вернама|)}

\section{Расстояние единственности}\label{section_unicity_distance}\index{расстояние единственности}
\selectlanguage{russian}
\index{расстояние единственности}

Использование ключей с длиной, сопоставимой с размером текста, имеет смысл только в очень редких случаях, когда есть возможность предварительно обменяться ключевой информацией большого объёма. Много большего, чем планируемый объём передаваемой информации. Но в большинстве случаев использование абсолютно надёжных систем оказывается неэффективным как с экономической, так и с практической точек зрения. Если двум сторонам нужно постоянно обмениваться большим объёмом информации и они смогли найти надёжный канал для передачи ключа, то ничего не мешает воспользоваться этим же каналом для передачи самой информации сопоставимого объёма.

В подавляющем большинстве криптосистем размер ключа много меньше размера открытого текста, который нужно передать. Попробуем оценить теоретическую надёжность подобных систем, исходя из статистических теоретико-информационных соображений.

Если длина ключа может быть много меньше длины открытого текста, то это означает, что энтропия ключа\index{энтропия!ключа} может быть много меньше энтропии открытого текста\index{энтропия!открытого текста}: $H(K) \ll H(M)$. Для таких ситуаций важным понятием является \textbf{расстояние единственности}\index{расстояние единственности}, впервые предложенное в работах Клода Шеннона~\cite{Golomb:2002, Schneier:2011}.

\begin{definition}\label{definition:unicity_distance}
\textbf{Расстоянием единственности}\index{расстояние единственности} называется количество символов шифротекста, которое необходимо для однозначного восстановления открытого текста.
\end{definition}

Пусть зашифрованное сообщение (шифротекст) $C$ состоит из $N$ символов $L$-буквенного алфавита:
	\[C = (C_1, C_2, \dots, C_N).\]

Определим функцию $h(n)$ как условную энтропию\index{энтропия!условная} ключа при перехвате криптоаналитиком $n$ символов шифротекста:
\[ \begin{array}{l}
    h ( 0 ) = H(K), \\
    h ( 1 ) = H(K | C_1), \\
    h ( 2 ) = H(K | C_1 C_2), \\
    \dots \\
    h ( n ) = H(K | C_1 C_2 \dots C_n), \\
    \dots
\end{array} \]

Функция $h(n)$ называется \emph{функцией неопределённости ключа}\index{функция!неопределённости ключа}. Она является невозрастающей функцией числа перехваченных символов $n$. Если для некоторого значения $n_u$ окажется, что $h ( n_u ) = 0$, то это будет означать, что ключ $K$ является детерминированной функцией первых $n_u$ символов шифротекста $C_1, C_2, \dots, C_{n_u}$, и при неограниченных вычислительных возможностях используемый ключ $K$ может быть определён. Число $n_u$ и будет являться \emph{расстоянием единственности}. Полученное $n_u$ соответствует определению~\ref{definition:unicity_distance}, так как для корректной криптосистемы однозначное определение ключа также означает и возможность получить открытый текст однозначным способом.

Найдём типичное поведение функции $h(n)$ и значение расстояния единственности $n_u$. Используем следующие предположения.
\begin{itemize}
    \item Криптограф всегда стремится спроектировать систему таким образом, чтобы символы шифрованного текста имели равномерное распределение и, следовательно, энтропия шифротекста\index{энтропия!шифротекста} имела максимальное значение:
            \[ H(C_1 C_2 \dots C_n) \approx n \log_2 L, ~ n = 1, 2, \dots, N. \]
    \item Имеет место соотношение
            \[ H(C | K) = H(C_1 C_2 \dots C_N | K)  =  H(M), \]
        которое следует из цепочки равенств
            \[ H(MCK) = H(M) + H(K | M) + H(C | MK) = H(M) + H(K), \]
        так как
            \[ H(K | M) = H(K), ~~ H(C | MK) = 0, \]
            \[H(MCK) = H(K) + H(C | K) + H(M | CK) = H(K) + H(C | K), \]
        поскольку
            \[ H(M | CK) = 0. \]
    \item Предполагается, что для любого $n \le N$ приближённо выполняется соотношение
        \[ H(C_n | K) \approx \frac{1}{N} H(M), \]
        \[ H(C_1 C_2\dots C_n | K) \approx \frac{n}{N} H(M). \]
\end{itemize}

Вычислим энтропию $H(C_1 C_2 \dots C_n ; K)$ двумя способами:
    \[ H( C_1 C_2 \dots C_n ; K ) = H(C_1 C_2 \dots C_n) + H(K | C_1 C_2 \dots C_n) \approx \]
        \[ \approx n \log_2 L + h(n), \]
    \[ H( C_1 C_2 \dots C_n ; K ) = H(K) + H(C_1 C_2 \dots C_n | K) \approx \]
        \[ \approx H(K) + \frac{n}{N} H(M). \]

Отсюда следует, что
    \[ h(n) \approx H(K) + n \left( \frac{H(M)}{N} - \log_2 L \right) \]
и
    \[ n_u = \frac{H(K)}{ \left( 1 - \frac{H(M)}{N \log_2 L} \right) \log_2 L} = \frac{H(K)}{\rho \log_2 L}. \]
Здесь
    \[ \rho = 1 - \frac{H(M)}{N \log_2 L} \]
означает избыточность источника открытых текстов\index{избыточность!открытого текста}.

Если избыточность источника измеряется в битах на символ, а ключ шифрования выбирается случайным образом из всего множества ключей $\{0, 1\}^{l_K}$, где $l_K$ -- длина ключа в битах, то расстояние единственности $n$ тоже получается в битах, и формула значительно упрощается:

\begin{equation}\label{eq:unicity_distance_simple_frac}
n_u \approx \frac{l_K}{\rho}.
\end{equation}

Взяв нижнюю границу $H(M)$ энтропии\index{энтропия!открытого текста} одного символа английского текста как $1{,}3$ бит/символ~\cite{Shannon:1951, Schneier:2002}, получим:

	\[ \rho _{en} \approx 1 - \frac{ 1{,}3 }{ \log _2 {26} } \approx 0{,}72.\]

Для русского текста с энтропией $H(M)$, примерно равной $3{,}01$ бит/символ~\cite{Lebedev:1958}\footnote{Следует отметить, что для английского текста значение $1{,}3$ представляет собой суммарную оценку для всего текста, в то время как оценка $3{,}01$ для русского текста получена Лебедевым и Гармашем из анализа \textbf{частот трёхбуквенных сочетаний} в отрывке текста Л.\,Н.\,Толстого <<Война и мир>> длиной в 30 тыс. символов. Соответствующая оценка для английского текста, также приведённая в работе Шеннона, примерно равна $3{,}0$.}, получаем:

	\[ \rho _{ru} \approx 1 - \frac{ 3{,}0 }{ \log _2 {32} } \approx 0{,}40.\]

Однако если предположить, что текст передаётся в формате простого текстового файла (\langen{plain text}) в стандартной кодировке UTF-8 (один байт на английский символ и два -- на кириллицу), то значения избыточности становятся примерно равны $0{,}83$ для английского и $0{,}81$ для русского языков:
\begin{gather*}
\rho _{en, \text{\textit{UTF-8}}} \approx 1 - \frac{ 1{,}3 }{ \log _2 {2^{8}} } \approx 0{,}83,\\
\rho _{ru, \text{\textit{UTF-8}}} \approx 1 - \frac{ 3{,}0 }{ \log _2 {2^{16}} } \approx 0{,}81.
\end{gather*} 

Подставляя полученные числа в выражение~\ref{eq:unicity_distance_simple_frac} для шифров DES\index{шифр!DES} и AES\index{шифр!AES}, получаем таблицу~\ref{table:unicity_distances}.

\begin{table}[!ht]
	\centering
		\begin{tabular}{|| l | r | r ||}
			\hline
			\hline
			\text{Блочный шифр} & \text{Английский текст} & \text{Русский текст} \\
			\hline
			\hline
			\text{Шифр DES\index{шифр!DES},} & \text{ $\approx~67$ бит;} & \text{$\approx~69$ бит;} \\
			\text{ключ 56 бит} & \text{ 2 блока данных} & \text{2 блока данных} \\
			\hline
			\text{Шифр AES\index{шифр!AES},} & \text{ $\approx~153$ бит;} & \text{$\approx~158$ бит;} \\
			\text{ключ 128 бит} & \text{ 3 блока данных} & \text{3 блока данных} \\
			\hline
			\hline
		\end{tabular}
  \caption{Расстояния единственности для шифров DES\index{шифр!DES} и AES\index{шифр!AES} для английского и русского текстов в формате простого текстового файла и кодировке UTF-8}
	\label{table:unicity_distances}
\end{table}

Полученные данные с теоретической точки зрения означают, что, когда криптоаналитик будет подбирать ключ к зашифрованным данным, трёх блоков данных ему будет достаточно, чтобы сделать вывод о правильности выбора ключа расшифрования и корректности дешифровки, если известно, что в качестве открытого текста выступает простой текстовый файл. Если открытым текстом является случайный набор данных, то криптоаналитик не сможет отличить правильно расшифрованный набор данных от неправильного и расстояние единственности, в соответствии с выводами выше (для нулевой избыточности источника), оказывается равным бесконечности.

Улучшить ситуацию для легального пользователя помогает предварительное сжатие открытого текста с помощью алгоритмов архивации, что уменьшает его избыточность\index{избыточность!открытого текста} (а также уменьшает размер и ускоряет процесс шифрования в целом). Однако расстояние единственности не становится бесконечным, так как в результате работы алгоритмов архивации присутствуют различные константные сигнатуры, а для многих текстов можно заранее предсказать примерные словари сжатия, которые будут записаны как часть открытого текста. Более того, используемые на практике программы безопасной передачи данных вынуждены, так или иначе, встраивать механизмы хотя бы частичной быстрой проверки правильности ключа расшифрования (например добавлением известной сигнатуры в начало открытого текста). Делается это для того, чтобы сообщить легальному получателю об ошибке ввода ключа, если такая ошибка случится.

Соображения выше показывают, что для одного ключа расшифрования, так или иначе, процедура проверки его корректности является быстрой. Чтобы значительно усложнить работу криптоаналитику, множество ключей, которые требуется перебрать, должно быть большой величиной (например от $2^{80}$). Это можно сделать, во-первых, увеличением битовой длины ключа, во-вторых, аккуратной разработкой алгоритма шифрования, чтобы криптоаналитик не смог <<отбросить>> часть ключей без их полной проверки.

Несмотря на то, что теоретический вывод о совершенной криптостойкости для практики неприемлем, так как требует большого объёма ключа, сравнимого с объёмом открытого текста, разработанные идеи находят успешное применение в современных криптосистемах. Вытекающий из идей Шеннона принцип выравнивания апостериорного распределения символов в шифротекстах используется в современных криптосистемах с помощью многократных итераций (раундов), включающих замены и перестановки.

